
%*****************************************
\chapter{Theoretical Foundation and Hypotheses}\label{ch:third}
%*****************************************

In the following chapter, we will present a scientific work which has not yet been published and which serves as the theoretical core of our thesis. This scientific paper goes by the title 
\begin{center}
 \textit{"Designing Efficient Authentication Mechanisms: There is More to Efficiency than Input Speed."}   
\end{center}
Thus the authors still remain unknown, we will refer to them as Anonymous et al. \cite{anonymous} in the following course of this thesis. As implied in Section \ref{2.2}, our intention for this chapter is to introduce an interesting contribution made towards making the first steps to setting a standard for the evaluation of efficiency in smartphone authentication mechanisms. We will begin by presenting the findings of Anonymous et al. \cite{anonymous}. Then, we will proceed by elaborating on the limitations and the incompleteness of their approach. Lastly, we will explain how we intend to validate their findings by reevaluating their approach and proving their hypotheses.\\

The intentions of Anonymous et al. \cite{anonymous} were to create a measurement standard for evaluating the usability of an authentication mechanism in terms of its efficiency. They noticed how several previous studies indicate that users commonly prefer authentication mechanisms which require little to no mental effort [QUELLE]. Moreover, they realized that those mechanisms were the ones to be rated most usable and fastest to use, by users. Researchers made these revelations possible, by categorizing the overall authentication time into \textbf{orientation} and \textbf{input} times. Anonymous et al. \cite{anonymous} analyzed the danger that occurred from only considering input times in authentication time measurements thus it can lead to false conclusions about the user-acceptance of authentication mechanisms. This is especially the case when mechanisms are set in comparison to each other. \\

In their approach, Anonymous et al. \cite{anonymous} began by reinterpreting the architecture of the general smartphone authentication process. [FIGURE]
They took its the structure and defined its period by determining a fix starting and ending point for it. The starting point was defined as the event in which the smartphone screen is turned on for the purpose of authentication. The ending point was set to be the second the last input action (i.e. button press/tap, node connection, etc.). Next, they subdivided this overall authentication period into the following phases \cite{anonymous}: 

\begin{itemize}
    \item \textbf{\textcolor{orange}{Orientation:}} In previous research, this phase has been commonly defined as the \textit{preparation} phase \cite{anonymous}. It defines the time period, beginning from the moment a smartphone screen is switched on, to them moment in which the first input action is made \cite{anonymous}. Orientation represents the time that users spend before entering their secret. This is the time which they spend either recalling their secret, preparing themselves for its input, or both \cite{anonymous}. It is considered to be the part of the authentication process which required the most mental effort.  
    \item \textbf{\textcolor{blue}{Active Authentication:}} This phase defines the time needed for entering the secret. It begins with the very first input action and ends with the very last. With last input action, we mean the moment after which the unlocking is either permitted or denied. We will call this phase the \textbf{input} phase, for simplicity reasons.
    \item \textbf{\textcolor{red}{Error Recovery:}} This phase occurs in situations where input errors occur. Its purpose is to signify the user that an error has occurred and to provide the possibility of recovering by making the user restart the authentication or by allowing so called \textit{undo operations}, which allow the user to correct their input and proceed with the authentication. 
    \item \textbf{\textcolor{green}{Clean Up:}} This phase is not considered to be a solid part of the authentication process, yet in some newly developed mechanisms it is crucial for completing the authentication. An example for its use is found in the concepts \textbf{TinyLock} by Kwon et al. \cite{kwon} and \textbf{Whispercore} by Airowaily et al. \cite{Airowaily}. In these designs, the clean up phase was intended to remove oily residues on the smartphone screen and thereby counteract the chance of potential smudge attacks \cite{anonymous}. 
\end{itemize}

The two phases which are considered to be highly essential are the orientation and input phases. In previous approaches, the input phase has often been considered to define the actual authentication procedure. Consequently, orientation time was often was often disregarded and ignored in usability evaluations. Error recovery is a phase that is not essentially mandatory for authentication, yet very useful for error management. Anonymous et al. \cite{anonymous} find that when implemented, it can have a significant effect on the efficiency of an authentication mechanism. In some cases, it is possible for its implementation to cause for further orientation or clean up phases \cite{anonymous}. This, in return, may cause for a longer authentication duration. The implementation of the clean up phase depends on the design of the authentication concept. Authentication is possible with of without it. \\

Through outlining the structure of an authentication process, Anonymous et al. \cite{anonymous} made a collection observations and factors which they wanted to regard and prove in a user case study. One factor to consider is the way in which authentication phases are proportioned because it has a major influence on the perceived efficiency of a mechanism. Moreover, the longer the orientation time of an authentication mechanism, the slower and less efficient it is perceived. In fact, cases in which the duration of the orientation phase exceeds the input phase, have been seen to be widely disliked by users. \\

Another factor that they wanted to incorporate is the "temporal arrangement of authentication phases" \cite{anonymous}. For instance, the authentication concepts Pattern Rotation \cite{Marbles} and Marbles highly differ in the temporal arrangement of their phase. While Pattern Rotation only has one orientation phase which precedes the input phase, Marbles has multiple small orientation and input phases, ordered in an alternating manner. Interestingly, Anonymous et al. \cite{anonymous} state that the latter has a possibility of decreasing the perceived duration of the authentication. Furthermore, they find that the coherence of orientation and input phases in terms of their context, have a major impact on a mechanisms perceived efficiency. Meaning, the less coherent the contexts of orientation time and input time are, the less efficient and convenient they are perceived. This factor is regarded in the design of Marbles \cite{Marbles}. Thus its use alternates from finding the marbles and entering them, the contexts of the tasks complement each other, Zezschwitz et al. \cite{Marbles} reported that study participants perceived it to be efficient. Moreover, through further research Anonymous et al. \cite{anonymous} have found that humans tend to perceive time periods longer than they are, if the contexts of these periods are incoherent \cite{anonymous,perception}.\\

Lastly, Anonymous et al. \cite{anonymous} states that the way in which error recovery is managed throughout the authentication of a mechanism, has an impact on its user-acceptance. In section [..], we discussed how users tended more towards Pattern authentication than Pin because it managed errors better. Despite the fact that Pattern took longer to authenticate with than Pin. This is because Pin allowed managed errors through undo-operations. These have shown to be disliked and therefore seldom used in designs. 


