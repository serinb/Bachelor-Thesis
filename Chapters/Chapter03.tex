
%*****************************************
\chapter{Theoretical Foundation and Hypotheses}\label{ch:third}
%*****************************************

In the following chapter, we will present an scientific work which is still to be published and which serves as the theoretical core of our thesis. This scientific paper is titled \textit{Designing Efficient Authentication Mechanisms: There is More to Efficiency than Input Speed}. Thus the authors still remain unknown, we will refer to them as Anonymous et al. \cite{anonymous} in the following course of this thesis. As implied in Section \ref{2.2}, our intention for this chapter is to introduce an interesting contribution made towards making the first steps to setting a standard for evaluation efficiency in smartphone authentication mechanisms. We will begin by presenting the findings of Anonymous et al. \cite{anonymous}. Then, we will proceed by elaborating on the limitations and the incompleteness of their approach. Lastly, we will explain how we intend to validate their findings by reevaluating their approach and proving their hypotheses.\\

The intentions of Anonymous et al. \cite{anonymous} were to create a measurement standard for evaluating the usability of an authentication mechanism in terms of its efficiency. They noticed how several previous studies indicate that users commonly prefer authentication mechanisms which require little to no mental effort [QUELLE]. Also, they also realized that those mechanisms were the ones to be rated most usable and fastest to use by users. These revelations were possible, by categorizing authentication time into \textbf{orientation} and \textbf{input} times. Anonymous et al. \cite{anonymous} noted the danger that came from only considering input times in authentication time measurements. Further research had shown that excluding \textbf{orientation} time from the measurement and evaluation lead to false conclusions about the usability of an authentication mechanism. The case was more severe when mechanisms were compared with each in this manner.  \\

First, Anonymous et al. \cite{anonymous} began by reinterpreting the structure of the general authentication concept. [FIGURE]
They 