
%*****************************************
\chapter{Theoretical Foundation and Hypotheses}\label{ch:third}
%*****************************************

Humans perception differs from reality. Psychological evidence in other fields ..\\

Here is where we will introduce the work on which the thesis is based on. \\
Maybe add a reference to some psychological /neurological research on time perception \\

[Zezschwitz] were able to present remarkable results through their research. They made their examinations by asking participants to use three different authentication concepts, each for a period of ten days. These, were the following: 

\begin{itemize}
    \item Android Pattern Unlock, which represents the "Short Orientation / Short Input" ratio
    \item Pattern Rotation, which represent the "long orientation / short input" ratio
    \item Marbles, which ration, where the orientation and input times are intertwined.
\end{itemize}

Through their study, they showed that users are more likely to prefer authentication systems that have a short orientation time which, therefore, appear less mentally demanding - As already mentioned in [CHAPTER].\par
In order to examine the validity of these findings even closer, we developed an interaction system as an application [CHAPTER REFERENCE] to study the overall affect of orientation time regarding perceived efficiency.  by analysing the following ratios: 

\begin{itemize}
    \item short orientation / short input, which was used as a baseline for the time measurement
    \item short orientation / long input (mainly examined phases)
    \item long orientation / short input
\end{itemize}

Since the study, conducted by [Zezschwitz], did not include the case for "Short Orientation / Long Orientation", we found it interesting to find out how the users would react to both main ratios, in comparison to each other. We did so, by presenting an interaction mechanism [READ CHAPTER] which emulated an authentication concept, presenting each of the mentioned ratios. In other words, we intended for our examination to be more focused and exact, by narrowing down the independent variable from "authentication concept" to "orientation and input time ratio".