
%*****************************************
\chapter{Theoretical Foundation and Hypotheses}\label{ch:third}
%*****************************************

In the following chapter, we will present an unpublished scientific work that will serve as the theoretical core of our thesis. This scientific paper goes by the title 
\begin{center}
 \textit{"Designing Efficient Authentication Mechanisms: There is More to Efficiency than Input Speed."}   
\end{center}
Thus the authors remain unknown. We will refer to them as Anonymous et al. \cite{anonymous} in the following course of this thesis. As implied in Section \ref{2.2}, our intention for this chapter is to introduce an interesting contribution made towards setting a standard for the evaluation of efficiency in smartphone authentication mechanisms. We will begin by presenting the findings of Anonymous et al. \cite{anonymous}. Then, we will proceed by elaborating on the limitations of their approach. Lastly, we will explain how we intend to validate their findings by reevaluating their approach and proving their hypotheses.

\section{Approach}

As discussed in Chapter \ref{ch:second}, the lack of usability in various authentication mechanisms, has been the main reason why many smartphone users refuse to use a screen lock for their phones. In Section \ref{2.2}, we saw how researchers attempted to make authentication mechanisms more usable by increasing their perceived effectiveness. Unfortunately, this approach was only successful to a certain extent and was not sufficiently productive for solving the usability issue. We learned that users generally value efficiency over effectiveness when it comes to authentication mechanisms and would prefer no screen lock rather than use one that is time-consuming and inconvenient (see Section \ref{2.2}). Consequently, researchers tried to detect the factors that most affected the efficiency of smartphone security and realized that these factors could not be assessed by simply measuring the duration of the input. They noticed a factor that had often been discarded during the evaluation of efficiency, and that is mental effort (see Section \ref{2.2}). The amount of mental effort that is needed to accomplish an authentication task is crucial for user-acceptance \cite{anonymous}. Therefore, new measurement methods are in need which approach authentication more on a human level and are designed with respect to humans' perception of time and cognitive abilities. \\

To that, Anonymous et al. \cite{anonymous} made an effort to create a measurement standard for evaluating the usability of authentication mechanisms in terms of their efficiency \cite{anonymous}. They noticed how previous studies indicated that users commonly prefer authentication mechanisms that require little to no mental effort \cite{anonymous, AnatomySmartphone}. Moreover, they realized that those mechanisms were the ones to be rated most usable and fastest to use. Researchers made these revelations possible, by categorizing the overall authentication time into \textbf{orientation} and \textbf{input} times \cite{anonymous}. Anonymous et al. \cite{anonymous} also analyzed the danger that occurred from only considering input times in evaluations. Thus it can lead to false conclusions about the efficiency of authentication mechanisms. This is especially the case when mechanisms are set in comparison to each other \cite{anonymous}. \\

Anonymous et al. \cite{anonymous} began by reinterpreting the architecture of the general smartphone authentication process. [FIGURE!!!] Next, they subdivided the overall authentication period into the following phases \cite{anonymous}: 

\begin{itemize}
    \item \textbf{\textcolor{orange}{Orientation:}} In previous research, this phrase has been commonly defined as the \textit{preparation} phase \cite{anonymous}. It defines the period, beginning from the moment a smartphone screen is switched on, to the moment when the first input action is made \cite{anonymous}. It usually takes place before the user enters their secret. This is the time which they spend either recalling their secret, preparing themselves for its input or both \cite{anonymous}. It is considered to be the part of the authentication process which required the most mental effort \cite{anonymous}.  
    \item \textbf{\textcolor{blue}{Active Authentication:}} This phase defines the time a user needs to enter their secret. It begins with the very first input action and ends with the very last. With the last input action, we mean the moment which determines whether unlocking is permitted or denied \cite{anonymous}. We will call this phase the \textbf{input} phase, for simplicity reasons.
    \item \textbf{\textcolor{red}{Error Recovery:}} This phase is useful in situations where the user makes an input error. Its purpose is to signify the user that an error has occurred and to provide the possibility of recovering from it by making the user restart the authentication or by allowing so called \textit{undo operations}, which allow the user to correct their mistake and proceed with the input. 
    \item \textbf{\textcolor{green}{Clean Up:}} This phase is not considered to be a solid part of the authentication process, yet in some newly developed mechanisms, it is crucial for completing the authentication. An example for its use, is found in the concepts \textbf{TinyLock} by Kwon et al. \cite{kwon} and \textbf{Whispercore} by Airowaily et al. \cite{Airowaily}. In these designs, the clean up phase was intended to remove oily residues on the smartphone screen and thereby counteract the chance of potential smudge attacks \cite{anonymous}. 
\end{itemize}

The two phases which are considered to be highly essential are the orientation and input phases. In previous approaches, the input phase has often been considered to define the actual authentication procedure. Consequently, orientation time was often disregarded and ignored in usability evaluations. Error recovery is a phase that is not essentially mandatory for authentication, yet very useful for error management. Anonymous et al. \cite{anonymous} find that its implementation can have a significant effect on the efficiency of an authentication mechanism. In some cases, its implementation can cause for further orientation or clean up phases \cite{anonymous}. This, in return, may cause a longer authentication duration. The implementation of the clean up phase depends on the design of the authentication concept. Authentication is possible with or without it. \\

By outlining the structure of the authentication process, Anonymous et al. \cite{anonymous} made a collection of observations and factors which they wanted to regard and prove in a user case study. First, they wanted to consider how authentication phases are proportioned. Reason being that it has a significant influence on the perceived efficiency of a mechanism. Moreover, the longer the orientation time of an authentication mechanism is, the slower and less efficient it is perceived. In fact, cases in which the duration of the orientation phase exceeds the input phase, have been seen to be widely disliked by users. \\

Second, that they wanted to consider how authentication phases are ordered in an authentication process \cite{anonymous}. Thus this factor may highly differ amongst authentication concepts, such as Pattern Rotation and Marbles \cite{Marbles} (Section \ref{2.2.3}), it is very important to regard. For instance, Pattern Rotation only consists of two phases, with the orientation preceding the input phase. Marbles, on the other hand, has multiple small orientation and input phases, ordered in an alternating manner. Interestingly, Anonymous et al. \cite{anonymous} find that the latter has the possibility of decreasing the perceived duration of authentication. \\

Third, they want to regard the coherence of orientation and input phases in terms of their contexts. Thus it also has a significant impact on a mechanism's perceived efficiency \cite{anonymous}. Meaning, the less coherent the contexts of orientation time and input time are, the less efficient and convenient they are perceived. This factor was regarded in the design of Marbles \cite{Marbles}. Thus its use alternates from finding the marbles and entering them, the contexts of the tasks complement each other. Through further research Anonymous et al. \cite{anonymous} have found that humans tend to perceive periods longer than they are if the contexts of these periods are incoherent \cite{anonymous,perception}.\\

Lastly, Anonymous et al. \cite{anonymous} state that error recovery should be cautiously managed throughout the authentication of a mechanism \cite{anonymous}. As mentioned above, the implementation of error recovery may result in further orientation and clean up phases. This observation was shown in findings by Zezschwitz et al. \cite{PatternWild} discussed in section \ref{2.2}. They discovered how users tended more towards Pattern authentication than Pin because it managed errors better. Even though Pattern took longer to authenticate with than Pin. This is because Pin allowed managed errors through undo-operations. These have shown to be disliked and therefore are seldom used in designs \cite{PatternWild, anonymous}. 

\section{User Case Study}

To prove their assumptions and observations, Anonymous et al. \cite{anonymous} decided to focus mainly on the phases, orientation, and input. Thus they are most important during authentication. To do so, they selected three authentication concepts, each representing a different ratio of orientation and input time \cite{anonymous}
\footnote{To better understand the functionalities of the concepts listed above, we recommend revisiting the approach by Zezschwitz et al. \cite{Marbles}, presented in Section \ref{2.2.3}.}: 

\begin{itemize}
    \item \textbf{Android Unlock Pattern} presented a \textcolor{red}{short} orientation - \textcolor{red}{short} input ratio,
    \item \textbf{Pattern Rotation} presented a \textcolor{blue}{long} orientation - \textcolor{red}{short} input ratio,
    \item \textbf{Marbles} presented a ratio in which orientation and input time were interlaced.
\end{itemize}

It is important to note that Pattern Rotation and Marbles were slightly modified in this study. Pattern Rotation presented a larger grid than the original design, and in Marbles, the elements (marbles) were small images rather than colors only [FIGURE !!!]. All three concepts were implemented in a prototype which was intended to be installed on the Android smartphones of study participants \cite{anonymous}. The prototype was intended to serve as an authentication system on the participants' phones. Each of the concepts was planned to be tested for ten days \cite{anonymous}. After each ten-day period, an online survey was required to be taken. Also, during the concept-tests, orientation and input times were logged for each authentication \cite{anonymous}. Orientation time was logged from the moment the screen was switched on, to the first input event. Input time was logged from the first to the last input event \cite{anonymous}. Participants were allowed to choose their own secrets for the concepts. However, patterns for Android Unlock Pattern had to consist of six nodes, patterns for Pattern Rotation had to have five nodes, and secrets in Marbles had to consist of four elements \cite{anonymous}.

\subsection{Results}

The study yielded 19 participants and delivered a set of 18 valid data entities \cite{anonymous}. Anonymous et al. \cite{anonymous} wanted to examine the different outcomes that result when authentication times are analysed differently. First, they analyzed the overall authentication times of the concepts and noted the following ranking in regarding the \textbf{measured performances}\footnote{The concepts are ordered from fastest to slowest (or best to worst) in this and the following rankings of this section.} \cite{anonymous}:

\begin{enumerate}
    \item \textbf{Android Unlock Pattern},
    \item \textbf{Pattern Rotation},
    \item \textbf{Marbles}.
\end{enumerate} 

However, when they analysed the input times of each other concepts only, they noticed the following difference:

\begin{enumerate}
    \item \textbf{Pattern Rotation},
    \item \textbf{Android Unlock Pattern},
    \item \textbf{Marbles}.
\end{enumerate}

Last, they analyzed the orientation time of the concepts and realized another significantly different outcome. Android Pattern Unlock required the least amount of mental effort and therefore had the shortest orientation time, on average \cite{anonymous}. Interestingly, there was hardly any difference between Pattern Rotation and Marbles, in terms of their average orientation times \cite{anonymous}. The \textbf{perceived efficiency} of the concepts was rated qualitatively though five-point likert scales \cite{anonymous}. Results showed that Android Unlock Pattern was seen as the fastest of all three concepts. More than half of the participants considered Marbles to be efficient, despite it being measured slower than Pattern Rotation. Moreover, half of the participants perceived Pattern Rotation as efficient \cite{anonymous}. Also, participants ranked the concepts in the following when asked if they contented a fast and easy orientation \cite{anonymous}: 

\begin{enumerate}
     \item \textbf{Android Unlock Pattern},
    \item \textbf{Marbles},
    \item \textbf{Pattern Rotation}.
\end{enumerate}

Lastly, when asked about the required cognitive effort, all participants approved of Android Pattern Unlock requiring the least amount of mental effort, followed by Pattern Rotation, then Marbles \cite{anonymous}.

\section{Limitations}

\section{Our Approach}

