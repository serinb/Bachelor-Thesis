
%*****************************************
\chapter{Theoretical Foundation and Hypotheses}\label{ch:third}
%*****************************************


The following chapter presents an ongoing scientific work of the \textit{Methods in Multi-Layer Usable Security Research} group at Rheinischische Friedrich-Wilhelms-Universit{\"a}t Bonn. It will serve as the theoretical core of our thesis. The scientific work goes by the title 
\begin{center}
 \textit{"Designing Efficient Authentication Mechanisms: There is More to Efficiency than Input Speed."}   
\end{center}

and the researchers who have contributed to it are Dr. Emanuel von Zezschwitz and Christian Tiefenau (M.Sc.). Throughout this thesis they will be referred to as Zezschwitz et al. \cite{Zezschwitz}.
First, we will present their findings and observations. Then we will discuss the limitations, which they faced throughout the study and accompany them with a collection of suggestive improvements, which were implemented in our complementary study in Chapter \ref{ch:fifth}.

\section{Approach}

As mentioned in Section \ref{2.3}, researchers have tried to detect the factors that most affect the efficiency of smartphone security. They realized that these factors could not be assessed by simply measuring the duration of authentication processes. Also, they discovered a factor that had often been disregarded during the evaluation of efficiency, and that is \textit{orientation time} (see Section \ref{2.2}). The amount preparation and mental effort that is needed to accomplish an authentication task is found to be crucial for user-acceptance \cite{Zezschwitz}. Therefore, new measurement methods are in need which approach the practice of authentication on a human level and which are designed with respect to humans' perception of time and their cognitive abilities. \\

To that, Zezschwitz et al. \cite{Zezschwitz} made an effort of examining a specific measurement method to better evaluate the usability of authentication mechanisms in terms of their perceived efficiency. They noticed how previous studies indicated that users commonly prefer authentication mechanisms that require little to no mental effort \cite{Zezschwitz, AnatomySmartphone}. So, they redefined the architecture of a general authentication process in order to detect the triggers that make authentication exhaustive and inconvenient for users. When found, they propose that their approach could be considered a prime step towards setting a standard for measuring usability in authentication mechanisms. Moreover, they suggest that their approach could help preventing the formation of false conclusions, in regard to the usability aspect of authentication mechanisms, especially when they are evaluated or compared to each other.  \\

\begin{figure}[t!]
\centering
\includegraphics[width=13cm, height=2cm]{Chapters/graphics/Phases.PNG}
\caption{Component phases of an authentication process. While Orientation and Active Authentication (input) phase are the core components of an authentication procedure, Error Recovery may also be included. Clean up phases are not considered part of the authentication process, yet in some designs they are crucial for its completeness \cite{Zezschwitz}. }
\label{fig:phases}
\end{figure}

The following list describes the phases that define the structure of an authentication process, according to Zezschwitz et al. \cite{Zezschwitz} (see Figure \ref{fig:phases}):

\begin{itemize}
    \item \textbf{\textcolor{orange}{Orientation:}} In previous research, this phase was commonly defined as the \textit{preparation} phase . It defines the period, beginning from the moment when a smartphone screen is switched on, to the moment when the first input action is made . It usually takes place before the user enters their secret. It is the time which they spend either recalling their secret, preparing themselves for its input, or both . Also, it is considered to be the part of the authentication process which requires the most mental effort.  
    \item \textbf{\textcolor{blue}{Active Authentication:}} This phase defines the time a user needs to enter their secret. It begins with the very first input action and ends with the very last\footnote{ \textit{Last input action} means the moment which determines whether unlocking is permitted or denied \cite{Zezschwitz}.}. This phase will be called the \textbf{input} phase, for simplicity reasons.
    \item \textbf{\textcolor{red}{Error Recovery:}} This phase is useful in situations in which an input error occurs. Its purpose is to signify to the user that an error had been made. The recovery can take place by requesting a restart of the authentication process, or by allowing so called \textit{undo operations}, which allow the user to correct their mistake and proceed with the input. 
    \item \textbf{\textcolor{green}{Clean Up:}} This phase is not considered to be a solid part of the authentication process, yet in some newly developed mechanisms, it is crucial for completing the authentication. An example for its use, is found in the concepts \textbf{TinyLock} by Kwon et al. \cite{kwon} and \textbf{Whispercore} by Airowaily et al. \cite{Airowaily}. In these designs, the clean up phase is intended to remove oily residues on the smartphone screen and thereby counteracts the chance for potential smudge attacks. 
\end{itemize}

According to Zezschwitz et al. \cite{Zezschwitz}, \textit{orientation} and \textit{input} phases are considered to be the most essential parts of an authentication process. In previous approaches, the \textit{input} phase has often been considered to define the actual authentication procedure. Consequently, \textit{orientation time} was often disregarded and ignored in usability evaluations. \textit{Error recovery} is not  considered to be essentially mandatory for successful authentication, yet it is very useful for its error management. They also observed that its implementation could have a significant effect on the efficiency of an authentication mechanism. In some cases, its implementation may even trigger the need for further \textit{orientation} or \textit{clean up} phases \cite{Zezschwitz}. This, in turn, may cause a longer authentication duration. The implementation of the \textit{clean up} phase depends on the design of the authentication concept. Authentication is possible with or without it (see Figure \ref{fig:phases}). \\

By outlining the structure of the authentication process, they proposed a collection of observations and factors to prove in a user case study. First, they observed that the proportioning of the authentication phases may affect the perceived efficiency of an authentication mechanism. For instance, they noticed that the longer the \textit{orientation} of an authentication mechanism is, the slower and less efficient it is perceived. In fact, cases in which the duration of the \textit{orientation phase} exceeded the \textit{input} phase, have been seen to be widely disliked by users. \\

Second, they noticed that the ordering of the phases might impact perceived efficiency. For instance, \textbf{Pattern Rotation} only consists of two phases, with the \textit{orientation phase} preceding the \textit{input phase}. \textbf{Marbles}, on the other hand, has multiple short \textit{orientation} and \textit{input phases}, which are ordered in an alternating manner (see Section \ref{2.2.3}). They suggest that the latter has the possibility of decreasing the perceived duration of authentication. \\

Third, they assume that the coherence of the \textit{orientation} and \textit{input phases} also has a potential impact on a mechanism's perceived efficiency \cite{Zezschwitz}. Meaning, the less coherent the contexts of \textit{orientation time} and \textit{input time} are, the less efficient and convenient they are perceived. This factor was regarded in the design of \textbf{Marbles} \cite{Marbles}. Thus its tasks alternate between finding the marbles and dragging them into the gap, their contexts complement each other. Through further research Zezschwitz et al. \cite{Zezschwitz} found that humans tend to perceive periods longer than they are, if they consist of many incoherent contexts \cite{Zezschwitz,perception}.\\

Lastly, Zezschwitz et al. \cite{Zezschwitz} added that \textit{error recovery} should be cautiously managed throughout an authentication process \cite{Zezschwitz}. As mentioned above, the implementation of \textit{error recovery} may result in further \textit{orientation} and \textit{clean up} phases. This observation was shown in findings by Zezschwitz et al. \cite{PatternWild}, discussed in Section \ref{2.2}. They discovered how users tended more towards \textit{pattern} authentication than \textit{pin} because they preferred its management of errors. \textit{Pin} manages the errors through \textit{undo-operations}, which have been shown to be disliked amongst users. Therefore, they are seldomly used in designs \cite{PatternWild, Zezschwitz}. 


\section{User Case Study}

\begin{figure}[t!]
\centering
\includegraphics[width=14cm, height=7cm]{Chapters/graphics/androidPatternMarble.PNG}
\caption{The concepts that Zezschwitz et al. \cite{Zezschwitz} used in their study: Android Pattern Unlock, baseline (Left); Pattern Rotation (Middle); Marbles (Right). Compare with Figure \ref{fig:marbles} to see how the concepts were modified in this study.}
\label{fig:android}
\end{figure}

Zezschwitz et al. \cite{Zezschwitz} decided to conduct a study in which they mainly focused on the phases, \textit{orientation}, and \textit{input}, as they are the most important phases of an authentication process. To prove their assumptions and observations, they selected three authentication concepts, each representing a different ratio of both phases \cite{Zezschwitz}\footnote{To better understand the functionalities of the concepts listed above, it is recommended to revisite the approach by Zezschwitz et al. \cite{Marbles}, presented in Section \ref{2.2.3}.}: 

\begin{itemize}
    \item \textbf{Android Pattern Unlock} represented a "\textcolor{red}{short} orientation/\textcolor{red}{short} input" ratio,
    \item \textbf{Pattern Rotation} represented a "\textcolor{blue}{long} orientation/\textcolor{red}{short} input" ratio,
    \item \textbf{Marbles} represented a ratio in which orientation and input time were interlaced.
\end{itemize}

It is important to note that \textit{Pattern Rotation} and \textit{Marbles} \cite{Marbles} were slightly modified in this study. \textit{Pattern Rotation} presented a larger grid than the original design, and in \textit{Marbles}, the elements (marbles) were small images rather than colored dots (see Figures \ref{fig:android} and \ref{fig:marbles}). All three concepts were implemented in a prototype which was then installed on the Android smartphones of study participants \cite{Zezschwitz}. The prototype was intended to serve as an authentication system on the participants' phones. Each of the concepts was planned to be tested for ten days \cite{Zezschwitz}. After each ten-day period, an online survey was required to be taken. Also, during the concept-tests, \textit{orientation} and \textit{input times} were logged for each authentication. \textit{Orientation time} was logged from the moment the screen was switched on, to the first input event. \textit{Input time} was logged from the first to the last input event. Participants were allowed to choose their own secrets for the concepts. However, fixed guidelines were set, for instance, patterns for \textit{Android Pattern Unlock} had to consist of six nodes, patterns for \textit{Pattern Rotation} had to consist of five nodes, and secrets in \textit{Marbles} had to consist of four elements \cite{Zezschwitz}.

\subsection{Results}

The study yielded 19 participants and delivered a set of 18 valid data entities \cite{Zezschwitz}. Zezschwitz et al. \cite{Zezschwitz} were interested in observing the different outcomes that would result, if they used three different approaches to evaluate the efficiency of all three authentication concepts. They began with analyzing the overall authentication times of the concepts and noted the following ranking regarding their \textbf{measured performances}\footnote{The concepts are ordered from fastest to slowest (or best to worst) in this and the following rankings of this section.} \cite{Zezschwitz}:

\begin{enumerate}
    \item \textbf{Android Pattern Unlock},
    \item \textbf{Pattern Rotation},
    \item \textbf{Marbles}.
\end{enumerate} 

Then, they only analyzed the measured \textit{input times} of the concepts and noted the following difference:

\begin{enumerate}
    \item \textbf{Pattern Rotation},
    \item \textbf{Android Pattern Unlock},
    \item \textbf{Marbles}.
\end{enumerate}

Last, they analyzed the measured \textit{orientation time} of the concepts and realized another significantly different outcome. \textit{Android Pattern Unlock} had the shortest \textit{orientation time} of all three concepts, as it required the least amount of mental effort. \textit{Pattern Rotation} and \textit{Marbles} did not notably differ from each other, in terms of their average \textit{orientation} times \cite{Zezschwitz}. \\


The perceived efficiency of the concepts was rated qualitatively through five-point Likert scales. Results showed that \textit{Android Pattern Unlock} was seen as the fastest of all three concepts. More than half of the participants considered \textit{Marbles} to be efficient, despite it having been measured slower than \textit{Pattern Rotation}. Moreover, half of the participants also perceived \textit{Pattern Rotation} as efficient. Participants were asked if the concepts contented a fast and easy \textit{orientation}. The results delivered the following ranking \cite{Zezschwitz}: 

\begin{enumerate}
     \item \textbf{Android Pattern Unlock},
    \item \textbf{Marbles},
    \item \textbf{Pattern Rotation}.
\end{enumerate}

Lastly, when asked about the required cognitive effort, all participants approved that \textit{Android Pattern Unlock} required the least amount of mental effort, followed by \textit{Pattern Rotation}, then \textit{Marbles} \cite{Zezschwitz}.

\subsection{Design Recommendations} \label{3.2.2}

Based on the results of their study, Zezschwitz et al. \cite{Zezschwitz} made a list of design recommendations to consider in the creation of authentication concepts. They are intended to optimize the usability of an authentication concept by regulating the following aspects \cite{Zezschwitz}:

\begin{itemize}
    \item \textbf{Recommendation 1:} \textit{"Measure all Stages"}\\
    By including \textit{orientation time} into the time measurements, the possibility of making false conclusions about a concept's usefulness could be prevented.
    \item \textbf{Recommendation 2:} \textit{"Keep Orientation Time Low"}\\
    As \textit{orientation times} are disliked by users, it is suggested to keep them as short as possible, as the effect of long \textit{orientation times} cannot be counteracted by short \textit{input times}.
    \item \textbf{Recommendation 3:} \textit{"Optimize the Ratio between Orientation Time and [Input Time]"} \\
    \textit{Orientation phases} should not be longer than the corresponding \textit{input phases}, because they are not well accepted by users, regardless of how fast or efficient a concept's performance is.
    \item \textbf{Recommendation 4:} \textit{"Avoid/Minimize Randomization"}\\ 
    Although randomization is often used as a countermeasure towards attacks (e.g., shoulder surfing), it impacts the perceived efficiency of a concept negatively and should, therefore, be avoided. Especially in between the tasks of an authentication procedure.
    \item \textbf{Recommendation 5:} \textit{"Measure Perceived Speed"}\\
    It is recommended that the performance of an authentication concept is not only assessed quantitatively, yet also qualitatively, in order to obtain information on how users' perceive certain aspects of the concept (e.g. \textit{orientation time}).  
    \item \textbf{Recommendation 6:} \textit{"Optimize Context Switches"}\\
    The tasks in an authentication concept should be coherent in terms of their contexts, otherwise they are not perceived as usable. 
    \item \textbf{Recommendation 7:} \textit{"Provide Efficient and Non-interrupting Error Recovery"}\\
    \textit{Error Recovery} should be designed to be as "fast" and simple as possible. A given example for good \textit{error recovery} is \textit{Android Pattern Unlock}, as its method is non-disturbing, thus errors are indicated through colors.  
\end{itemize}


\section{Limitations and suggestive improvements} \label{3.3}

We will first begin by presenting some of the limitations of the study. Next, observations on specific qualities of the study will be presented and suggestive improvements on these qualities will be discussed. 

According to Zezschwitz et al. \cite{Zezschwitz}, the overall perception of the tested concepts could have been influenced by the participants' general preference \cite{Zezschwitz} because a person's culture and history have shown to influence their acceptance of a particular system \cite{Harbach:2016} (Section \ref{2.2.1}). The complementary study, presented in Chapter \ref{ch:fifth} intends to exclude this limitation by developing a single concept in which the studied ratios are represented. That way, a more genuine evaluation of the ratios should be possible, without any interference of participants' preferences regarding a particular concept. Moreover, Zezschwitz et al. \cite{Zezschwitz} stated that the measured times for \textit{orientation} might have differed from the actual times. Reason being, that the measurements for the \textit{orientation times} began as soon as the smartphone screen was turned on. However, not every screen activation was always immediately followed by an authentication \cite{Zezschwitz}. A possible solution for rectifying this inaccuracy, is to isolate the \textit{orientation phase} from any other possible action. The concept for the complementary study was designed in a way which required the user to actively initiate the authentication process. That way, a more fix and accurate starting point for the \textit{orientation time} could be defined.\\

Further aspects of the study, which we observed, are the following: The ratios chosen by Zezschwitz et al. \cite{Zezschwitz} might not have been suitable enough to receive a definite result on whether users truly prefer short \textit{orientation time} over long \textit{orientation time}. We found that it would be interesting to observe the outcome of testing ratios that have the same temporal length, yet are contrasting regarding the length of their phases. Thus not included by Zezschwitz et al., we considered to include the ratio \textit{"short orientation/long input"} into the complementary study. We assume that by mainly focusing on the ratios \textit{"long orientation/short input"} and \textit{"short orientation/long input"}, and by setting the ratio \textit{"short orientation/short input"} as a baseline, we could receive more detailed results on users' attitude towards different lengths of \textit{orientation phases}. A further suggestion is to observe whether different lengths of \textit{input phases} might play a role in users' preference and perception of efficiency. Lastly, we assume that by allowing study participants to qualitatively evaluate the ratios in comparison to each other, we could receive more detailed and precise information about which ratio is generally perceived as efficient and which is not. \\

The next chapter of this thesis will present a concept which was specifically designed for the sake of the complementary study, later presented in Chapter \ref{ch:fifth}. It was created specially designed to represent each of the previously mentioned ratios. Its sole purpose was to serve as supportive tool in the study and no further. Nonetheless it is important, for the scope of this thesis, to understand the design choices and the thought processes that were involved in creating it, as they were made from a user-centered design approach and also included HCI principles, to assure the creation of a system that is suitable for the desired goals and intentions of our study. 







