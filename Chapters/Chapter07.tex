
%*****************************************
\chapter{Conclusion and Future Work}\label{ch:seventh}
%*****************************************

The aim of this thesis, was to analyze how far the effects of certain factors, more specifically those of \textit{orientation time}, might influence the perceived efficiency, and therefore the usability of smartphone authentication concepts. We based our assumptions and hypotheses on the ongoing study which was presented in Chapter \ref{ch:third}. We took into consideration, a selection of improvements and approaches to provide proof for the observations and recommendations made by Zezschwitz et al. \cite{Zezschwitz}. One of the improvements, was to represent all studied phase ratios in a single concept. The intention, behind this approach, was to obtain more precise and authentic opinions on the ratios, without having the representative concepts be an influencing factor on participants' preferences. Moreover, we decided to observe a pair of contrasting ratios (\textit{long/short} and \textit{short/long}) to observe the true effect that \textit{orientation time} has on the user-acceptance of a concept, with regard to its \textit{input time}. A quantitative evaluation of the times for each ratio, allowed us to assure that the ratio implementations were suitable for the purpose of our study. Measurements showed that both contrasting ratios were close to the baseline and that, in the worst case, their contrasting phases did not substantially differ in length. This aspect granted us the possibility to compare participants' perception of both examined phases: \textit{orientation} and \textit{input}. A qualitative evaluation showed that participants were more sensitive towards the length of \textit{orientation time} than they were towards the length of input time. Therefore, they were able to estimate the duration of \textit{orientation} more correctly than the duration of \textit{input}. In summary, it could be said that \textit{short/long} was rated more positively than \textit{long/short}, as the majority of the participants agreed that it was more easy, efficient and that it required less mental effort. Our findings imply that, in general, users are less aggravated by long \textit{input times} than they are by long \textit{orientation times}. Moreover, they show that pairing short \textit{input phase} with long \textit{orientation phases} does not equalize the amount of mental effort needed for the ratio. Therefore, it is true that authentication concepts should consist of \textit{orientation times} which exceed the length of their corresponding \textit{input times}. This means that \textit{orientation times} should truly be kept as short as possible. As the implementation of our concept \underline{\textbf{FiPa}} contained a randomization feature\footnote{Thus overall, all grids in the application differed from each other, and mental tasks were complicated through uniquely set \textit{traps} (see Section \ref{4.2.2.3}).}, which was especially noticeable in the ratio \textit{long/short}, it caused for longer orientation times. For that, randomization should be reduced as much as possible, to contribute towards increasing the perceived efficiency of an authentication concept. However, as randomization is generally used as a measure to increase the security mechanisms, this means that newer countermeasures are called for which, simultaneously, agree with the usability of a concept and also enhance its security feature. Moreover, we were able to confirm that users generally prefer short \textit{orientation times} and that they are less bothered by longer \textit{input times}, by discovering that the majority of our participants favored to use the ratio \textit{short/long} for the smartphone's screen lock. \\

All in all, it is safe to say that our findings and the outcomes of our study represent further steps towards understanding how users understand and perceive the efficiency of authentication mechanisms. In addition, our discoveries have taken us one step closer towards creating a standard for the design of more efficient authentication mechanisms, as users' general likes and dislikes have been confirmed. In fact, our study has also drawn attention towards certain aspects which could also be considered in future research contributions. For instance, although we have found that users, generally prefer long \textit{input times}, we can not conclude that they are completely satisfied with them, regardless their length and complexity. As mentioned in Chapter \ref{ch:second}, the correct and secure use of password authentication mechanisms still remains an issue that has not yet been solved. Therefore the true limits of what users consider efficient, regarding \textit{input time} should be specifically examined and, if necessary, alternative solutions should be considered. Furthermore, in order to thoroughly assess the universality of our observations, it would be interesting to conduct a longitude study, using a similar design approach to ours, with participants who do not use a screen lock on their smartphone. That way, one could examine whether our complementary findings remain true over longer periods of use and whether they have the ability of improving the security behaviours and habits of users. Depending on the outcomes of this proposed study, it would then be possible to even create a provisional design standard for authentication mechanisms. That way, current and newly developed authentication concepts could be compared and evaluated with more certainty and validity. Consequently, one could observe whether preset and improved design standards of authentication mechanisms successfully apply in all real-life situations and cases. Thus our results indicated that users' preference of a particular concept might depend on their current mood.  

