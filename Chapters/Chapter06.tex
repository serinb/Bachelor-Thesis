
%*****************************************
\chapter{Discussion and Limitations}\label{ch:sixth}
%*****************************************

In this chapter, we will review and analyze the main results of the study (see Chapter \ref{ch:fifth}), in relation to the recommendations proposed in Section \ref{3.2.2}. Based on these results, we will discuss the universality of Zezschwitz et al.'s \cite{Zezschwitz} approach. Then, the limitations, which were faced during the design and procedure of our study, will be presented.\\

Analogous to Zezschwitz et al. \cite{Zezschwitz}, the main phases (\textit{orientation} and \textit{input}) were measured in the study. An improved approach was taken towards enhancing the accuracy of the time measurements, specifically those of \textit{orientation time}. Thus one of the limitations of their approach was the inaccuracy of the \textit{orientation time} measurements. The concept \underline{\textbf{FiPa}} was specially designed to to prevent this issue. Based on a further recommendation, the measurement of perceived efficiency was taken into consideration for the design of this study. Participants' perception of efficiency, regarding the ratios, was measured and assessed through a questionnaire. They were asked to directly compare the two main ratios \textit{short/long} and \textit{long/short} to each other, in terms of certain characteristics like mental effort, efficiency, preference, length of input, length of search and so on. \\

As explained earlier, the contrasting ratios \textit{short/long} and \textit{long/short} were designed to be asymmetrical in terms of their phase lengths\footnote{Meaning "long orientation" had to have the same length as "long input" and "short orientation" had to have the same length as "short input".}. The quantitative results showed that, on average, their overall length differed by 1176 ms. Thereby, \textit{long/short} had the longest average duration of 6056 ms.
Nonetheless, in the worst case, both ratios were relatively similar, regarding their long phases and their short phases. This shows that the design of the concept and the implementation of the ratios were suitable and successful for the purpose of our study. We were, therefore, able to prove our assumptions and observations (see Chapter \ref{ch:third}), thus our used method and the resulting quantitative data, were similar to how we had desired for them to be. \\

In terms of perceived efficiency, participants were first asked to evaluate the \textbf{mental} and \textbf{practical tasks} of each ratio, separately. They were asked to rate the annoyance of each ratio's \textit{orientation} and the simplicity to memorize as well as input its corresponding pattern. Interestingly, the majority of the participants found, both, short and long patterns easy to memorize and to input (see Figure \ref{fig:likert}). However, the difference between both ratios was notable regarding the annoyance of their \textit{orientation times}. While all participants disagreed that \textit{short orientation} was annoying, the average opinion regarding the annoyance of \textit{long orientation} was "neutral". Even if participants did not strongly incline towards a positive or negative opinion regarding the annoyance of \textit{long orientation}, the results show a greater acceptance of \textit{short orientation}. \\

As mentioned earlier, participants were asked to compare both ratios, based on certain characteristics. The majority of the participants (68\%) found that \textit{long/short} required more mental effort than \textit{short/long} (only 16\%). This implies that, although both ratios contained a complicated task\footnote{The "long phases" of each ratio were meant to represent complicated tasks.}, complicated \textbf{mental tasks} (long \textit{orientation times}) seem to have stressed participants more than complicated \textbf{practical tasks} (long \textit{input times}). Consequently, most participants (79\%) agreed that the implementation of \textit{short/long} was easier than \textit{long/short} (only 16\%). In addition, when participants were asked which of both ratios took longer to find and enter, the results indicated that participants were more sensitive towards the complexity of long \textit{orientation phases} than they were towards \textit{long input phases}. While all nineteen participants agreed that the pattern in \textit{long/short} took longer to find than in \textit{short/long}, only 74\% agreed that the pattern in \textit{short/long} took longer to enter. Through comparing participants' time estimates to the measured data, it was notable to see that they were right 89\% of the time regarding the ratio which had the longest \textit{orientation time}. However, less participants (57\%) estimated correctly, regarding the ratio which had the longest \textit{input time}. This shows that participants were more sensitive towards the time and mental effort needed to undergo the \textit{orientation phase}. Thus less of them were able to distinguish between the length of short and long \textit{input time}. Participants' incline towards the ratio \textit{short/long} was more notable when they were asked which ratio implementation they found to be more efficient. The majority leaned towards \textit{short/long} (58\%) and only 26\% found \textit{long/short} to be more efficient.\\

Based on these analyzed results, one might expect that, in general,  participants would have had a greater incline towards \textit{short/long} than towards \textit{long/short}. However, when asked to chose one ratio implementation as a screen lock for their smartphone, this was not the case. Although 47\% chose the ratio \textit{short/long} for their screen lock, still 42\% chose \textit{long/short}. It is clear to see that there is not a significant difference regarding the preference of both ratios. However, we assume that participants who chose \textit{long/short}, made their choice with regard to the potential security aspects of the ratio design, as the most frequently given reasons for \textit{long/short} were "more secure" and "more difficult". Nonetheless, most participants chose \textit{short/long} for their screen lock. The most frequent reasons given were "more secure", as well as "more efficient". These reasons imply a higher user-acceptance and preference for \textit{short/long}, in terms of usability. The remaining 11\% (2) did not choose either of both ratios, yet they gave interesting reasons why. One participant said that they preferred \textit{short/long}, as it is easier. Yet, because it would be less secure, they would much rather prefer a combination of both ratios for their screen lock. This implies that they would favor a concept that is not only easier to use, but that also assures to be secure enough to protect their smartphone. Another participant stated that it did not matter to them which ratio they chose, as they would get used to either ratio, with time. They added that their choice would depend, more specifically, on their mood. This opinion indicates that, that in some cases, the usability of a concept is not only dependent on its design and implementation, yet could also be determined by the user himself and by the specific needs and preferences they might have in a particular situation. It would be interesting to thoroughly analyze this observation in the future.\\

To see whether there was a correlation between participants' ratio preferences and their personal screen lock choice, they were asked to share the type of authentication mechanism which they used and the length or complexity of their secret. We assumed that participants who used short and simple secrets, would choose \textit{short/long}, and the ones who used longer secrets would choose the latter. However, we could not find any notable behavioral patterns. \\

All in all, the conducted study has shown that by "measuring all stages", it was possible to receive a more accurate conception of each ratios' measured performance. Combined with qualitatively "measuring perceived speed", participants' perceptions and opinions appeared more comprehensible and justifiable during the evaluation \cite{Zezschwitz}. By comparing participants' perception on the implementation of \textit{long/short} and \textit{short/long}, it was noticeable that the majority of participants favored the concept design, where \textit{orientation time} did not exceed \textit{input time}. Although \textit{long/short} had a simple \textbf{practical task}, results showed that it did not compensate for its complicated \textbf{mental task}. This validates previous observations in the project which indicated that "\textit{orientation time} should be kept as low as possible" for better user-acceptance \cite{Zezschwitz}.

The fact that more participants found \textit{long orientation} to be longer than \textit{long input}, validates that phase ratios should be "in favor" of the input phase. Reason being that users are less bothered by the required effort for accomplishing a complicated \textbf{practical task}, than they are for accomplishing a  complicated \textbf{mental tasks}. Especially when mental tasks are randomized. In our concept \underline{\textbf{FiPa}}, each grid was unique for all ratio implementations, yet the factor of randomization was more noticeable for \textit{long/short}. The \textit{traps}, which were uniquely set, complicated the search process even more. This is also a reason why participants had a stronger incline towards the orientation of \textit{short/long}. This means that the inclusion of randomization in authentication concept designs is disliked by users, as it causes for longer \textit{orientation times}. For that, it should be avoided or minimized as much as possible, as suggested by Zezschwitz et al. \cite{Zezschwitz}.

\section{Limitations}

Although \underline{\textbf{FiPa}} was designed to emulate an authentication process in a lab setting, it is not clear whether participants' evaluation of the ratios might have differed if they had interacted with the concept in real-life scenarios\footnote{Meaning if the participants had used the implementation of the concept as an authentication mechanism for a certain period of time (analogous to the study design of Zezschwitz et al. \cite{Zezschwitz}).}. As the concept was only used for a short period of time, there is a chance that participants' preferences and the study results might have differed.\\

Unfortunately, the study had to be conducted twice. The first study involved a wider and more versatile demographic, however, its quantitative results were not usable, by cause of an error in the implementation of the time measurements of the application. Sadly, the error was detected, afterwards, during the evaluation of the results. Consequently, an additional study had to be conducted and a new set of participants had to be recruited because the former participants were already familiar with the contents of the study and would have caused a bias in the results. The majority of the newly recruited participants were computer science students, as they had to be collected on short notice. This is another reason why the results of this study cannot be fully generalized, as most of the participants had an IT-background and were also familiar with the importance of smartphone security. The outcome of the study would have been more interesting and convincing, if our participants had a wider demographic and did not use screen locks on their smartphone. In combination with a longitudinal study, it would have been possible to obtain more detailed information on the perceived efficiency and the user-acceptance of the main ratios, represented in our concept. As mentioned in Chapter \ref{ch:second}, the primary reason why smartphone users persist not to use a screen lock is due to their perceived inconvenience. It would have been interesting to find out, which of the ratio implementation had the potential of motivating users to use authentication mechanisms on their smartphone and improve their smartphone security behavior. Depending on their preference regarding the ratios it would have been possible to derive the true effect of \textit{orientation time} on perceived efficiency. As a result a standard could be developed, to guarantee an enhanced user-acceptance and a higher perceived efficiency for future authentication concept propositions. \\

In the study, the questionnaire included a semantic differential with the intent to evaluate the aesthetics of the application. The intention was to examine how participants perceived the aesthetics of the application and to see whether their performance and preference was positively or negatively influenced by it. However, during the evaluation it was noted that these observations would exceed the scope of our thesis and distract from our main focus, namely the effect of \textit{orientation time} on the perceived efficiency of authentication mechanisms.\\

We would like to note that our choice to measure the times for every third level of each phase, in the application, was not based on a study. Instead, we examined the quantitative data and calculated the amount of input errors made amongst all participants, for each level separately. We found that, in total, only 4 input errors were made in the first level, followed by 6 input errors in the second level and only 2 input errors in the third.\\

Lastly, we would like to mention that one recommendation remains, which we were not able to validate. That is, "Optimize Context Switches" (see Section \ref{3.2.2}). Although the \textbf{mental} and \textbf{practical tasks} in \underline{\textbf{FiPa}} were designed to be coherent and were intended complement each other, we did not lay focus on letting participants evaluate this feature of the concept.
