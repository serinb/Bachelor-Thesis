
%*****************************************
\chapter{Discussion and Limitations}\label{ch:sixth}
%*****************************************

In this chapter, the main results of the study (see Chapter \ref{ch:fifth}) will be reviewed and analyzed, in relation to the recommendations proposed by Anonymous et al. \cite{anonymous} (see Section \ref{3.2.2}). Based on these results the universality of Anonymous et al.'s \cite{anonymous} approach will be discussed. Last, the limitations which were faced during the design and procedure of the study, will be presented.\\

Analogous to Anonymous et al. \cite{anonymous}, the main phases (\textit{orientation} and \textit{input}) were measured in the study. An improved approach was taken towards enhancing the accuracy of the time measurements, specifically those of \textit{orientation time}. Thus one of the limitations in Anonymous et al.'s \cite{anonymous} approach, was the inaccuracy of the \textit{orientation time} measurements. The concept \underline{\textbf{FiPa}} was specially designed to implement Anonymous et al.'s \cite{anonymous} measurement method, to prevent this issue. \\
Based on a further recommendation by Anonymous et al. \cite{anonymous} the measurement of perceived efficiency was taken into consideration for the design of the study. Participants' perception of efficiency, regarding the ratios, was measured and assessed through a questionnaire. They were asked to directly compare the two main ratios \textit{short/long} and \textit{long/short} to each other, in terms of certain characteristics (mental effort, efficiency, preference, length of input, length of search, etc.). \\

Although the contrasting ratios \textit{short/long} and \textit{long/short} were designed to be symmetrical in terms of their phase lengths\footnote{Meaning long orientation had to have the same length as long input and short orientation had to have the same length as short input.}, quantitative results have shown that, on average, their overall length differed by 1176 ms, with \textit{long/short} having the longest average duration (6056 ms). Nonetheless, apart from \textit{long/short's} the maximum orientation time (9053 ms), its remaining \textit{orientation times} did not exceed 5202 ms. In reference to the \textit{orientation times} of \textit{short/long}, this means that the "long phases" of both ratios did not differ substantially, and that their worst-case measurements were similar. The same is true for the maximum times of the "short phases" in both ratios: "short input" ( in \textit{long/short}) and "short orientation" (in \textit{short/long}) barely differed by 8ms. This shows that the design of the concept and the implementation of the ratios were suitable and successful for the purpose of this study, as in their worst cases they can be considered almost equally efficient, in terms of their measured efficiency. \\

In terms of perceived efficiency, participants were first asked to evaluate the \textbf{mental} and \textbf{practical tasks} of each ratio, separately. They were asked to rate the annoyance of a ratio's \textit{orientation} and the simplicity to memorize, as well as input its corresponding pattern. Surprisingly the memorization and input of the short and long patterns, were considered easy to memorize and input, by all participants, on average. However, the difference between both ratios was notable regarding the annoyance of their \textit{orientation times}. While, on average, all participants disagreed that \textit{short orientation} was annoying, the average opinion regarding the annoyance of \textit{long orientation} was "neutral". Even if participants did not strongly incline towards a positive or negative opinion regarding the annoyance of \textit{long orientation}, the results show a greater acceptance of \textit{short orientation}.\\

As mentioned earlier, participants were asked to compare both ratios, based on certain characteristics. The majority of the participants (68\%) found that \textit{long/short} required more mental effort than \textit{short/long} (only 16\%). This implies that, although both ratios contained a complicated task\footnote{The "long phases" of each ratio were meant to represent complicated tasks.}, complicated \textbf{mental tasks} seem to have stressed participants more than complicated \textbf{practical tasks}. Consequently, most participants (79\%) agreed that the implementation of \textit{short/long} was easier than \textit{long/short} (only 16\%). In addition, when participants were asked which of both ratios took longer to find and enter, the results indicated that participants were more sensitive towards the complexity of long \textit{orientation phases} than they were towards \textit{long input phases}. While all participants (100\%) agreed that the pattern in \textit{long/short} took longer to find than in \textit{short/long}, only 74\% agreed that it took longer to enter in \textit{short/long} and the rest (26\%) saw no difference. Participants' incline towards the ratio \textit{short/long} was more notable when they were asked which ratio implementation they found more efficient. The majority leaned towards \textit{short/long} and only 26\% found \textit{long/short} to be more efficient. \\

Based on these analyzed results, one might expect that, in general,  participants would have had a greater incline towards \textit{short/long} than towards \textit{long/short}. However, when asked to chose one ratio implementation as a screen lock for their smartphone, this was not the case. Although 47\% chose the ratio \textit{short/long} for their screen lock, still 42\% chose \textit{long/short}. It is clear to see that there isn't a significant difference between both percentile rates, as the only differ by 5\%. It is assumed that participants who chose \textit{long/short}, made their choice with regard to the potential security aspects of the ratio design, as the most frequently given reasons for \textit{long/short} were "more secure" and "more difficult". Nonetheless, most participants chose \textit{short/long} for their screen lock. The most frequent reasons given were "more secure", as well as "more efficient". These reasons imply a higher user-acceptance and preference for \textit{short/long}, in terms of usability. The remaining 11\% (2) did not choose either of both ratios, yet they gave interesting reasons why. One participant said that they preferred \textit{short/long}, as it is easier. Yet, because it would be less secure, they would much rather prefer a combination of both ratios for their screen lock. This implies they would favor a concept that is not only easier to use, but that also assures to be secure enough to protect their smartphone. Another participant stated that it did not matter to them which one, they chose, as they would get used to either ratio, with time. They added that the choice would depend, more specifically on one's mood. This opinion indicates that, for some, the usability of a concept is not only dependent of its design and implementation, yet could also be determined by the user himself and by the specific needs and preferences they might have in a particular situation. \\

To see whether there was a correlation between participants' ratio preferences and their personal screen lock choice, they were asked to share the type of authentication mechanism which they used and the length or complexity of their secret. It was assumed that participants who used short and simple secrets, would choose \textit{short/long}, and the ones who used longer secrets would choose the latter. In comparing both sets of information, no notable behavioral pattern was noted. \\

All in all, the conducted study has shown that by "measuring all stages" \cite{anonymous}, it was possible to receive a more accurate conception of the ratios' measured performance. Combined with "measuring perceived speed" \cite{anonymous} qualitatively, participants' perceptions and opinion appeared more comprehensible and justifiable during the evaluation. By comparing participants' perception on the implementation of \textit{long/short} and \textit{short/long}, it was noticeable that majority participants favored the concept design, where \textit{orientation time} did not exceed \textit{input time}. Although \textit{long/short} had an simple \textbf{practical task}, results showed that it did not compensate for complicated \textbf{mental task} which it presented \cite{anonymous}. This validates Anonymous et al.'s \cite{anonymous} observation that \textit{orientation time} should be kept low for better user-acceptance. The fact that more participants found \textit{long orientation} to be longer than \textit{long input}, validates that phase ratios should be "in favor" of the input phase \cite{anonymous}. This shows that users are less bothered by the required effort for accomplishing a complicated \textbf{practical tasks}, than they are of complicated \textbf{mental tasks}. Especially when mental tasks are randomized, as each one represents a unique challenge, which can not be simplified over time through the strength of muscle memory (as with memorizing long secrets). In the concept \underline{\textbf{FiPa}}, each grid differed from the other in all ratio implementations, yet the factor of randomization was more noticeable for \textit{long/short}, as the \textit{traps}, which were uniquely set, complicated the search process even more. This is also a reason why participants had a stronger incline towards the orientation of \textit{short/long}. Therefore, randomization should truly be avoided or minimized, as suggested by Anonymous et al. \cite{anonymous}. \\


\section{Limitations}

Although \underline{\textbf{FiPa}} was designed to emulate an authentication process, to help participants adjust to the context of the study more easily, it is not clear whether their evaluation of the ratios might have differed if the they had interacted with the concept in real-life scenarios\footnote{Meaning if the participants had used the implementation of the concept as an authentication mechanism for a certain period of time (similar to the study design of Anonymous et al. \cite{anonymous}).}. As the concept was only used for a short period of time, there is a chance that participants' preferences and the study results might have differed.\\

Unfortunately, the study had to be conducted twice. The first study involved a wider and more versatile demographic, however, its quantitative results were not usable, by cause of an error in the implementation of the time measurements of the application. Sadly, the error was detected, afterwards, during the evaluation of the results. Consequently, an additional study had to be conducted and a new set of participants had to be recruited because the former participants were already familiar with the contents of the study and would have caused a bias in the results. The majority of the newly recruited participants were computer science students, as they had to be collected on short notice. This is another reason why the results of this study cannot be fully generalized, as most of the participants had an IT-background and were also familiar with the importance of smartphone security. The outcome of the study would have been more interesting and convincing, if the participants had a wider demographic and did not use screen locks on their smartphone. In combination with a longitude study, it would have been possible to obtain more detailed information on the perceived efficiency and the user-acceptance of the main ratios, represented in the concept. As mentioned in Chapter \ref{ch:second}, the primary reason why smartphone users persist not to use a screen lock is due to their perceived inconvenience. It would have been interesting to find out, which of the ratio implementation had the potential of motivating users to use authentication mechanisms on their smartphone and improve their smartphone security behavior. Depending on their preference regarding the ratios it would have been possible to derive the true effect of \textit{orientation time} and perceived efficiency. As a result a standard could be developed, to guarantee a enhanced user-acceptance and a higher perceived efficiency for future authentication concept propositions.

In the study, the questionnaire included a semantic differential, with the intent to evaluate the aesthetics of the application. The intention was to examine how participants perceived the aesthetics of the application and to see whether their performance and preference was positively or negatively influenced by it. However, during the evaluation it was noted that these observations would exceed the scope of this thesis and distract from its true focus, namely the effect of orientation time on the perceived efficiency of authentication mechanisms.  
