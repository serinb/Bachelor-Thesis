
%*****************************************
\chapter{Discussion}\label{ch:sixth}
%*****************************************

In this chapter we will assess the analysis of our results on some recommendations made by Anonymous et al. \cite{anonymous}. We will then include some observations from our study which we found interesting. In the end we will elaborate on the limitations which we faced during work.

\section{Analysis}

We will provide a list of the recommendations by Anonymous et al. \cite{anonymous}, presented in chapter \ref{ch:third}, for a better understanding of our references throughout this section: 

\begin{itemize}
    \item \textbf{Recommendation 1:} \textit{"Measure all Stages"}
    \item \textbf{Recommendation 2:} \textit{"Keep Orientation Time Low"}
    \item \textbf{Recommendation 3:} \textit{"Optimize the Ration between Orientation Time and [Input Time]"} 
    \item \textbf{Recommendation 4:} \textit{"Avoid/Minimize Randomization"}
    \item \textbf{Recommendation 5:} \textit{"Measure Perceived Speed"}
    \item \textbf{Recommendation 6:} \textit{"Optimize Context Switches"}
    \item \textbf{Recommendation 7:} \textit{"Provide Efficient and Non-interrupting Error Recovery"}
\end{itemize}



\textbf{Recommendation 1:} \textit{"Measure all Stages"}\\
Anonymous et al. \cite{anonymous} recommend that by including orientation time specifically into our measurements, we can prevent setting false conclusions about the usefulness of an authentication concept. 

This suggestion was confirmed through our precise measurement of the combinations. Had we only measured the input times of the combinations, we would come to false conclusion that \textit{Long/Short} is faster and therefor more efficient than \textit{Short/Long}. Moreover, the qualitative results would have appeared incomprehensible and contradicting, thus orientation, which had a large impact on the participants perception, was disregarded. \\

\textbf{Recommendation 4:} \textit{"Measure Perceived Efficiency"} \\
Ananymous et al. proposed that by qualitatively measuring the performance of a system, one could assess the impact that orientation time on its acceptance. \\

We noticed by comparing quantitative and qualitative data to each other, that participants' time estimation deviated from the actual time measurements to a certain extent. For instance, all participants shared the opinion that \textit{Long/Short's} pattern took longer to find. Although this outcome may seem obvious, given the complexity of combination. However, differences were noticeable regarding other aspects. Although the combination \textit{Short/Long} had a significantly longer pattern than \textit{Long/Short}, 26\% did not see a difference between both combinations' input lengths. Moreover, in terms of the overall duration of the combinations, more than half of the participants misestimated.



\textbf{Recommendation 2:} \textit{"Keep Orientation Time Low"} \\
Anonymous et al. \cite{anonymous} state that orientation times is often perceived as annoying. Also the say that "short input times cannot compensate for high orientation" \cite{anonymous}.

When we asked our participants whether they found the searching process of \textit{Long/Short} annoying, the answers were balanced. The number f participants that found it annoying was the same as the number of participants that did not find it annoying. Yet, all participants shared the opinion that the search process of \textit{Short/Long} was not annoying. Although we can not validate that long orientation truly is perceived as annoying, there is a clear indication that short orientation is highly accepted. \\

\textbf{Recommendation 3:} \textit{"Optimize the Ration between Orientation Time and [Input Time]"} \\
Anonymous et al. \cite{anonymous} state that concepts that feature an orientation phase which exceeds the respective input phase, are have a poor user-acceptance. 

When we asked our users which of the two main combinations they would pick as a screen lock for their smartphones, 42\% chose \textit{Long/Short} and 47\% chose \textit{Short/Long}. Although the percentages do not significantly differ from each other, the reasons for the participant's choices serve as an indication of their genuine preference. Both parties selected \textit{security} as one of their reasons.
Therefore, we will focus on the other reasons given. The participants that chose the combination \textit{Long/Short} were more focused on the \textit{difficulty} aspect of the combination. We assume this is because they found that \textit{Short/Long} would be insecure. The one's that chose \textit{Short/Long} claimed it was more \textit{efficient}. We would like to note that 79\% of all participants found \textit{Short/Long} to be easier and 57\% found it more efficient. This implicates that the participants that chose \textit{Long/Short} as their imaginative screen lock prioritized security over convenience. This is perceivable through the additional reasons given by some of the participants. One of the chose \textit{Long/Short} although she found \textit{Short/Long} to be more \textit{comfortable}. Moreover, two participants said they chose \textit{Short/Long} because it would be easier to use at times when they are tired. Another participant chose neither of both combinations yet showed a preference towards \textit{Short/Long} because of its ease-of-use. Therefore, it is clear that in terms of use, \textit{Short/Long} had a greater acceptance. We would also like to note that our results show a slight correlation between what users perceive as \textit{easy} and what the perceive as \textit{efficient}. In addition, because the majority claimed that \textit{Long/Short} required more mental effort, we can conclude that long orientation times, especially the ones that exceed the respective input time, are poorly accepted by users. \\







Most of the users had screen locks on their personal phones. It would be more interesting to create a long study in which the participants are known to not comply with smartphone authentication and see whether or not they prefer the concept that requires little orientation. Then contact them some time later and see if they changed to using a screen lock.


Thus the majority of our participants were Computer Science students we can not generalize the findings to be exact for everybody , thus they might be more security conscious than others and therefore do not care much about usability. Our first study included a wider range of demographic yet unfortunately our data could no longer be used. in order to truly validate these findings it would be reasonable to purposely select a wider range of demographics such that the number of IT savy paticipants equals the amount of non IT savy people.  \\


Although slightly beyond the scope of our thesis, we wanted to find out whether the aesthetics of our prototype had an influence on our participants' performances. To be exact we wanted to examine whether a positive appeal of the prototype's might of had a positive effect on the user's performance, and vice versa. The results were assessed using a semantic differential scale [REFER TO APPENDIX]. (see whether the participants that had very long/short times evaluated the aesthetics bad or good.  //


We were not able to validate avoiding randomization, because we only used one concept and randomization into it (like marbles) would have made it even more difficult to use. 

