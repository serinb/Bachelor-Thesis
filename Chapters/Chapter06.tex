
%*****************************************
\chapter{Discussion}\label{ch:sixth}
%*****************************************

Most of the users had screen locks on their personal phones. It would be more interesting to create a long study in which the participants are known to not comply with smartphone authentication and see whether or not they prefer the concept that requires little orientation. Then contact them some time later and see if they changed to using a screen lock.


Thus the majority of our participants were Computer Science students we can not generalize the findings to be exact for everybody , thus they might be more security conscious than others and therefore do not care much about usability. Our first study included a wider range of demographic yet unfortunately our data could no longer be used. in order to truly validate these findings it would be reasonable to purposely select a wider range of demographics such that the number of IT savy paticipants equals the amount of non IT savy people.  \\


Although slightly beyond the scope of our thesis, we wanted to find out whether the aesthetics of our prototype had an influence on our participants' performances. To be exact we wanted to examine whether a positive appeal of the prototype's might of had a positive effect on the user's performance, and vice versa. The results were assessed using a semantic differential scale [REFER TO APPENDIX]. (see whether the participants that had very long/short times evaluated the aesthetics bad or good.  


