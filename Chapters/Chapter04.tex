
%*****************************************
\chapter{Developing an Interaction System}\label{ch:forth}
%*****************************************
ADD SHORT INTRODUCTION TO THE CHAPTER !!\\

Before entering into the design process, it is crucial to mention that the following interaction system was created solely to serve as a tool in the user case study, which we will present in Chapter 5. The system is in no means a suggestion for an authentication concept and was not intended to be utilized as such. It's sole utilization and purpose was meant for the scope of this thesis and no further. Therefore certain aspects such as security and effectiveness have consciously not been considered during the design and the development of this system. Nonetheless, during its creation, we tried our best to follow the steps of a User-Centered Design (UCD) approach and included a selection of Human Computer Interaction (HCI) principles, in order to make the system easier to understand and function. It was important to make sure that the design and structure choices, neither distracted the user from the important factors of the study, nor complicated the effort needed to concentrate on them.  


\section{Requirements of the Concept}
As mentioned above, the interaction system was intended to be utilized as a tool to help examine certain factors later on in the study. These factors were \textbf{orientation time} and \textbf{input time}, previously presented in Chapter 3. In order to do so, the interaction system had to be based on a certain concept. This concept had to be divided into two coherent and related miniature tasks: a \textbf{mental task} and a \textbf{practical task}. The intention behind this division was to be able to measure the orientation time (duration of the mental task) and the input time (duration of the practical task), separately. Therefore, a built-in timer was required to undertake the measurements, which would be automatically saved in a database. 

In order to validate the findings presented in the previous chapter, we had to find a way through which we could examine the true effect that orientation time had on the perceived efficiency of an authentication concept, with respect to the input time. We, therefore, decided to analyse the following two contrasting time ratios: 
\begin{center}
    \textbf{Long} orientation - \textbf{Short} input \\
    versus \\
    \textbf{Short} orientation - \textbf{Long} input
\end{center} 

To have a baseline to which we could later compare our quantitative data during evaluation, we included a third ratio, where orientation time and input time were equally long:  
\begin{center}
\textbf{Short} orientation - \textbf{Short} input
\end{center} 

In contrast to the study by Anonymous et al. \cite{anonymous}, we wanted to directly examine the influence of orientation time, by comparing both contrasting variations against each other, using only one concept, instead of three[REFER TO CHAPTER]. We hoped that by unifying all variations into one concept, we could eliminate the possibility that our future participants' preferences could be influenced by their personal preference of a particular concept. Therefore, the concept had to be malleable in a way, such that the time needed to accomplish the miniature tasks, could be modified by adjusting their degree of difficulty and complexity. We figured that the less difficult and the less complex a task was, the lesser time it needs to be accomplished, and vice versa. This idea applied to both, mental and practical task.

\section{Concept Development}

In the following section, we will present the many design and evaluation attempts that were necessary to generate \underline{\textbf{FiPa}}: the interaction system presented and utilized in our study. 

\subsection{Fundamental Concept Idea}
We were interested in creating a concept that activated an interaction which believably resembled a smartphone authentication mechanism. We believed that by making the interaction similar to an authentication process, we could help our future participants better understand and adapt to the context of our study. That way, we hoped to receive more focused and detailed qualitative results. One of our concerns was to construct the concept as usable as possible, in order to shine the light on the true usability issues that we intended to examine. We, therefore, decided to create a graphic-based concept \cite{AnatomySmartphone}. Although recent research has shown them to take longer to authenticate and to be more error-prone, than alphanumeric approaches, they were still perceived as most user-accepted and easier to use, on average \cite{PatternWild}.\\

To better understand the basic idea of our concept, we recommend referring to Figure [ADD FIGURE HERE], whilst reading our explanation: 
The name of our concept, \underline{\textbf{FiPa}}, is inspired by its functionality. \underline{\textbf{FiPa}} is abbreviation for the phrase "\underline{\textbf{Fi}}nd \underline{\textbf{Pa}}ttern" and its meaning will be comprehensible through the following explanation of its procedure: \\
First, a predefined pattern is presented that has to be memorized well by the user. This pattern consists of a certain combination of buttons. Each button has a certain characteristic, that makes it distinguishable from the others. Memorizing the order of the buttons and their distinct characteristics is crucial for the both mental and practical task. After \underline{memorizing} the pattern, a large grid, filled with buttons, is presented. The \textbf{mental task} thereby is to \underline{find} the memorized pattern, hidden in the grid. When found, the \textbf{practical task} is to \underline{input} the found pattern correctly, as memorized. [REFER TO IMAGE HERE]\\

We derived the idea of using a grid of buttons from the concept Pattern Rotation \cite{patternRotation, anonymous}. The choice of using certain characteristics for each button was inspired by the concept Marbles \cite{patternRotation, anonymous}. We  tried to limit the amount of mental effort required for the interaction with the concept by making a few intentional design choices. We knew from the beginning, that in the practical task, the user had to make a kind of input contribution, so that we could attain the input time. Also, we were aware that the patterns, intended for memorization, had a complex structure. Thus not only the length of the pattern and the order of the buttons had to be learnt, yet also the characteristics of each button. On that note, we imagined that having to completely reproduce the memorized pattern for the input, would be tiresome and not possible for every user. Furthermore, we took into consideration that the user would not memorize the pattern permanently, meaning that it would most certainly be stored in their short-term memory. Taking into account that human beings have different memory spans[ADD SOURCE], we assumed that searching for the pattern would require less mental effort than its reproduction. That way, when the user stumbles upon the hidden pattern in the grid, they are more likely it spot it, because they recognizing it. 

\subsection{Concept Design}
In the following we will present the design approaches that lead us to creating and developing the interaction system \underline{\textbf{FiPa}}.

\subsubsection{First Draft}
Our initial vision for the design is illustrated in [ REFER TO EMOJI IMAGE HERE]. As in Marbles, we decided to make the different elements of the concept (the buttons) distinguishable through small images, emojis to be exact. We will first begin by explaining how we visualized the practical task and then proceed with explaining our intentions regarding the mental task. As mentioned in Section 4.2.1, a certain pattern is required to be memorized at the very beginning of the activity. To mark the beginning of a pattern we determined that the first button of each pattern should contain a key-emoji [REFER TO IMAGE ]. We will call this button the key-button. In order to modify whether the intended input time should be short or long, we created examples of how we imagined the patterns would look like in Figure [ADD IMAGE HERE]. As for the mental task, we envisioned to design a grid filled with emoji-buttons. The design of the grid depended on whether long orientation time or short orientation time was intended. As mentioned earlier in the chapter, we assumed that for a long orientation time we need to create a difficult and complex search process. We imagined that by incorporating multiple key-buttons, besides the one belonging to the hidden pattern, we could overwhelm the user and thereby complicate and elongate the search process. In contrast, we imagined it would be possible to facilitate the pattern search, by having the only key-button in the grid belong to the hidden pattern. That way the user could spot the pattern much easier and quicker.

\subsubsection{Evaluation: First Draft}
In order to find out if our concept was effective in practice, we created a paper prototype. The prototype presented the three ratios, introduced in Section 4.1. We were able to have six participants evaluate the prototype. It was important to us, to see whether or not our design negatively affected the usability of our concept. We wanted to assure that it was easy to use and easy to understand.
Five of the participants considered the many emojis to be overwhelming on the eyes and that they made grid appear very crowded. Moreover, the long pattern [FIGURE] was considered too complicated and was hard to memorize by all of the participants. 
Luckily, the idea behind the concept was liked by all of the participants. Especially, the notion of starting each pattern with a specific key-button. \\

The information gained and the lessons learnt through the previous evaluation phase, gave us a closer insight on the human's perception and cognitive ability. At this point, we were one step closer to creating a more usable interaction system.

\subsubsection{Second Draft}

!!! This section is still in progress !!! \\

For our second draft we decided to make a few adjustments regarding the overall aesthetics and the design of the concept. We decided to reduce the crowded appeal of the previous draft by replacing the emojis with three simple shapes: circle, square and triangle. Reason for sticking to images rather than letter or number is because humans were shoen to memorise images easier than numbers/letters...[add source] We also choose to incorporate a specific set of colors: red, green, blue and yellow. The reason for choosing these colors is ...[add source]. We kept the notion of each pattern beginning with a specific button and determined for this button to be red and contain a triangle. Reason being that red is an alerting color and the triangle... [add source]. We assumed that through the users preattentive perception they would be easily be able to spot the triangle in the grid and would be even more drawn to though its alerting color. Whilst designing the patterns we noticed that the size and the design of the grid had to also be taken into account. 


