
%*****************************************
\chapter{Developing an Interaction System}\label{ch:forth}
%*****************************************


Before entering into the design process it is crucial to mention that the following interaction system was created solely to serve as a tool in the user case study, which will be presented in Chapter 5. The use of this system was not intended for any use outside of the contributions for this thesis. Therefore certain aspects such as security and effectiveness have not been considered during the design and the development of this system. Nonetheless, during its creation, we tried our best to follow the steps of a User-Centered Design (UCD) approach and included a selection of Human Computer Interaction (HCI) principles, in order to make the system easier on the human eyes and on the mind. It was important to make sure that the design and structure choices, neither distracted the user from the important factors of the study, nor complicated the effort needed to concentrate on them.  

\section{Requirements of the design for the study}
As mentioned above, the interaction system was intended to be utilized to help examine certain factors later on in the study. These factors were \textbf{orientation time} and \textbf{input time}, previously explained in Chapter 3. In order to do so, the interaction system had to be based on a certain concept. This concept had to be divided into two coherent and related miniature tasks: a mental task and a practical task. The intention behind this division is to measure the overall time needed for the accomplishment, thereby distinguishing between the duration of the mental task (orientation time) and the duration of the practical task (input time). \\

In order to validate the findings presented in the previous chapter, we had to find a way through which we could examine the effect of orientation time on authentication, with respect of the input time. Therefore we decided to analyse two contrasting time ratios: 
\begin{center}
    \textbf{Long} orientation - \textbf{Short} input \\
    vs. \\
    \textbf{Short} orientation - \textbf{Long} input
\end{center} 

In contrast to the study by Anonymous et al. \cite{anonymous}, we wanted to directly examine the influence that orientation time has on users' acceptance of an authentication concept, by comparing both contrasting variations against each other, using only one concept, instead of three. Therefore the concept had to be malleable in a way, such that the time needed to accomplish the miniature tasks, could be modified by adjusting their degree of difficulty and complexity. We imagined the less difficult and the less complex the task, the less time is needed for its accomplishment, and vice versa. This approach applied the mental task, as well as the practical task. 

Another crucial requirement of the design, was that it included a built-in timer, that would measure the duration of the tasks separately and thereby deliver us the orientation and input needed.  

\section{Concept Development}






