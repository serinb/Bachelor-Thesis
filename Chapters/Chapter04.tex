
%*****************************************
\chapter{Developing an Interaction System}\label{ch:forth}
%*****************************************
\textbf{VERY LAST: }INTRODUCTION TO THE CHAPTER !!\\
Mentaiong that the interaction mechanism was inteneded to be a Android smartphone application.\\

Before entering into the design process, it is crucial to mention that the following interaction system was created solely to serve as a tool in the user case study, which we will present in Chapter \ref{ch:fifth}. The system is in no means a suggestion for an authentication concept and was not intended to be utilized as such. It's sole utilization and purpose was meant for the scope of this thesis and no further. Therefore certain aspects such as security and effectiveness have consciously not been considered during the design and the development of this system. Nonetheless, during its creation, we tried our best to follow the steps of a \textbf{User-Centered Design} (UCD) approach and included a selection of \textbf{Human Computer Interaction} (HCI) principles, in order to make the system easier to understand and function. It was important to make sure that our design and structure choices, neither distracted our future study participants from the important factors of the study, nor complicated the effort needed to concentrate on them.  


\section{Requirements of the Concept} \label{4.1}
As mentioned above, the interaction system was intended to be utilized as a tool to help examine certain factors later on in the study. These factors were \textbf{orientation time} and \textbf{input time}, previously presented in Chapter \ref{ch:third}. In order to do so, the interaction system had to be based on a certain concept. This concept had to be divided into two coherent and related miniature tasks: a \textbf{mental task} and a \textbf{practical task}. The intention behind this division was to be able to measure the \textbf{orientation time} \underline{(duration of the mental task)} and the \textbf{input time} \underline{(duration of the practical task)}, separately. Therefore, a built-in timer was required to undertake the measurements, which would be automatically saved in a database. \\

In order to validate the findings presented in Chapter \ref{ch:third}, we had to find a way through which we could examine the true effect that \textbf{orientation time} had on the perceived efficiency of an authentication concept, with respect to the \textbf{input time}. We, therefore, decided to analyse the following two contrasting time ratios: 
\begin{center}
    \textcolor{blue}{Long} orientation - \textcolor{red}{Short} input \\
    \textbf{vs.} \\
    \textcolor{red}{Short} orientation - \textcolor{blue}{Long} input.
\end{center} 

To have a baseline to which we could later compare our quantitative data during evaluation, we included a third ratio, where orientation time and input time were equally long. We call this:  
\begin{center}
\textcolor{red}{Short} orientation - \textcolor{red}{Short} input.
\end{center} 

In contrast to the study by Anonymous et al. \cite{anonymous}, we wanted to directly examine the influence of \textbf{orientation time}, by comparing both contrasting variations against each other, using only one concept, instead of three (Chapter \ref{ch:third}). We hoped that by unifying all variations into one concept, we could eliminate the possibility that our future participants' perceptions could be influenced by their personal preference of a particular concept. Therefore, the concept had to be malleable in a way, such that the time needed to accomplish the miniature tasks, could be modified by adjusting their degree of difficulty and complexity. We figured that the less difficult and the less complex a task was, the lesser time it needs to be accomplished, and vice versa. This assumption applied to both, \textbf{mental} and \textbf{practical task}.

\section{Concept Development} \label{4.2}

In the following section, we will present the design and evaluation attempts that were necessary to generate \underline{\textbf{FiPa}}: the interaction system presented and utilized in our study. 

\subsection{Fundamental Concept Idea} \label{4.2.1}
We were interested in creating a concept that activated an interaction which believably resembled a smartphone authentication. We believed that by making the interaction similar to an authentication process, we could help our future participants better understand and adapt to the context of our study. That way, we hoped to receive more focused and detailed qualitative results. One of our concerns was to construct the concept as usable as possible, in order to shine the light on the true usability issues which we intended to examine. We, therefore, decided to create a graphic-based concept. Although recent research has shown them to take longer to authenticate and to be more error-prone, than alphanumeric approaches \cite{AnatomySmartphone}, they were still perceived as most user-accepted and were considered easier to use, on average \cite{PatternWild}.\\

To better understand the basic idea of our concept, we recommend referring to Figure [ADD FIGURE HERE], whilst reading our explanation: 
The name of our concept, \underline{\textbf{FiPa}}, is inspired by its functionality. \underline{\textbf{FiPa}} is abbreviation for the phrase "\underline{\textbf{Fi}}nd \underline{\textbf{Pa}}ttern" and its meaning will be comprehensible through the following description of its procedure: \\
First, a predefined pattern is presented that has to be memorized well by the user. This pattern consists of a certain combination of buttons. Each button has a certain characteristic, that makes it distinguishable from others. Memorizing the order of the buttons and their distinct characteristics is crucial for the both \textbf{mental} and \textbf{practical task} (Section \ref{4.1}). After \underline{memorizing} the pattern, a large grid, filled with buttons, is presented. The \textbf{mental task} thereby is to \underline{find} the memorized pattern, hidden in the grid. When found, the \textbf{practical task} is to \underline{input} the found pattern correctly, as memorized. [REFER TO IMAGE HERE]\\

We derived the idea of using a grid of buttons from the concept \textit{Pattern Rotation} \cite{patternRotation, anonymous}. The choice of using certain characteristics for each button was inspired by the concept \textit{Marbles} \cite{patternRotation, anonymous}. We tried to limit the amount of mental effort required for interacting with the concept by making a few intentional design choices. We knew from the beginning, that in the \textbf{practical task}, we wanted the user to make an input, so that we could attain the \textbf{input time}. As mentioned above, the memorization process was composed of retaining the length of the pattern, the order of the buttons and their distinct characteristics. For that reason, we imagined that having to completely reproduce the memorized pattern for the input, would be tiresome and not possible for every user. We also took into consideration that the user would only retain the memorized as long as necessary for the study. This means that the pattern would most certainly be stored in their short-term memory. To that, we assumed that searching for the pattern would require less mental effort than its complete reproduction. That way, during the mental task, when the user stumbles upon the hidden pattern in the grid, they are more likely it spot it, because they recognize it. 

\subsection{Concept Design} \label{4.2.2}
In the following we will present the design approaches that lead us to creating and developing the interaction system \underline{\textbf{FiPa}}.

\subsubsection{First Draft} \label{4.2.2.1}
Our initial vision for the design is illustrated in [ REFER TO EMOJI IMAGE HERE]. As in \textit{Marbles} \cite{patternRotation, anonymous}, we decided to make the different elements of our concept (the buttons) distinguishable through small images, emojis to be exact. We will first begin by explaining how we visualized the \textbf{practical task} and then proceed with explaining our intentions regarding the \textbf{mental task}. As mentioned in Section 4.2.1, a certain pattern is required to be memorized at the very beginning of the activity. To mark the beginning of a pattern we determined the first button of each pattern to contain a key-emoji [REFER TO IMAGE ]. We will call this button the \textbf{key-button}. In order to modify whether the intended \textbf{input time} should be short or long, we created examples of how we imagined the patterns would look like in Figure [ADD IMAGE HERE]. Similar to the input method in \textit{Android Unlock Pattern}, we decided the user would enter the pattern by connecting the buttons with their finger. We found this method to be more suitable, being that \underline{\textbf{FiPa}} was intended to be a smartphone application and that the user would interact with it through a touch screen. \\

As for the \textbf{mental task}, we envisioned to design a grid filled with emoji-buttons. The design of the grid depended on whether long \textbf{orientation time} or short \textbf{orientation time} was intended. As mentioned earlier in the chapter, we assumed that for a long \textbf{orientation time} we need to create a difficult and complex search process. We imagined that by incorporating multiple \textit{key-buttons}, besides the one belonging to the hidden pattern, we could complicate and elongate the search process. In contrast, we imagined it would be possible to facilitate the pattern search, by having the only \textit{key-button} in the grid belong to the hidden pattern. That way the user could spot the pattern much easier and quicker.

\subsubsection{Evaluation: First Draft} \label{4.2.2.2}
In order to find out if our concept was effective in practice, we created a paper prototype. The prototype presented the three ratios (Section 4.1) and was composed of three patterns (two short and one long) and three grids (two with only one \textit{key-button} and one with multiple \textit{key-buttons}). We were able to gather six participants to evaluate our prototype. It was important to us, to see whether or not our design negatively affected the usability of our concept. We wanted to assure that it was easy to use and easy to understand. 

We shortly explained the functionality of our prototype, prior to the interaction.
Five of the participants considered the many emojis to be overwhelming on the eyes. They said that it made the grid appear very crowded. Moreover, the long pattern [FIGURE] was considered too complicated and was hard to memorize by all of the participants. Regarding the input method, we noticed that it was not suitable for the different \textbf{input time} variations that we wanted to achieve. Although it worked well for "short \textbf{input time}, the time needed to enter the long pattern was not as long as we had intended for it to be. 
Luckily, the idea behind the concept was liked by all of the participants. Especially, the notion of starting each pattern with a specific \textit{key-button}. Despite the flaws mentioned above, they understood the basic functioning of the concept well. \\

The information gained and the lessons learnt through the previous evaluation phase, gave us a closer insight on humans' perception and cognitive ability. At this point, we were one step closer to creating a more usable interaction system.

\subsubsection{Second Draft} \label{4.2.2.3}

For our second draft, we decided to make a few adjustments regarding the overall aesthetics and the design of the concept. We decided to reduce the crowded appeal of the previous draft by replacing the emojis with three simple shapes: \textit{circle}, \textit{square} and \textit{triangle}. Through further research, we learnt that due to the pictorial superiority effect \cite{pictorial}, humans are able to retain information through images, much better than through letters or numbers \cite{pictorial, 2014}. That is why we chose to still represent the characteristics of the buttons through shapes. Our initial choice of colors for the design was: \textit{red}, \textit{green} and \textit{blue}. The reason for our choice is that they are the basic colors that the human eye perceives naturally \cite{Butz2014}. We kept the notion of each pattern beginning with a specific button and decided for this button to be red containing a triangle [FIGURE]. We chose the triangle because it is a symbol that has been shown to convey the meaning of power\footnote{http://www.whiteriverdesign.com/meaning-shapes-design/ - last accessed: 2019/11/16} and permanence \cite{Frutiger1998}.  Moreover, the human eye is immediately attracted to its shape \footnote{https://designshack.net/articles/layouts/the-sometimes-hidden-meaning-of-shapes/ - last accessed: 2019/11/16}. We assumed that through the mentioned affect of the triangle and through the alerting feature of the color red, we could attract the user's pre-attentive perception to the buttons. \\

Next, we proceeded with the pattern and grid design. After a couple of trials, we determined the size of the grid to be 4x7, because we found that the size was big enough to hold a decent amount of buttons and yet suitable enough to not overly confuse the user during the interaction. Also, we decided to incorporate so called \textit{traps} into grids. \textit{Traps} are a set of buttons, that have a similar constellation and set of characteristics as the hidden pattern, yet are not identical to it. Their purpose was to mislead the user, during the \textbf{mental task} and thereby elongate the searching process, as explained in Fig [ADD IMAGE]. Another modification made, was the input method. We found that by pressing the buttons in the right order, rather than connecting them (Section \ref{4.2.2.1}) we could gain control of the approximate duration for the input even better. While creating the patterns, it was crucial that they did not take up too much space in the grid. That way, we could shuffle the buttons and set \textit{traps} much more freely. 

\subsubsection{Evaluation: Second Draft} \label{4.2.2.4}

As in our first draft evaluation (Section \ref{4.2.2.2}), we created a paper prototype to evaluate the changes and improvements that we made in our concept. However this time, the prototype was created differently. It was a  We  form of \textit{Wizard Of Oz} prototyping \cite{Butz2014}. During the interaction, we would uncover the succeeding event of the prototype, depending on the participant's "input" or "action". We wanted to test a certain structure to see if it would be suitable for the implementation of our concept. Thus, we have three ratios (Section \ref{4.1}) to examine, the prototype was structured accordingly into three parts. In the previous draft, we had only one mental and practical task per ratio. For simplicity, we will call the composition of mental and practical, a combination. Instead of having only one combination per ratio, we decided it would be reasonable to assign three for each ratio. That way, the user would have the chance to interact with each of the ratio-designs more than once and thereby remember them better during the qualitative segment of the study. An example for the structure of the paper prototype is illustrated in [FIGURE]. We tested our prototype with the same set of participants with whom we had tested our first draft. We did so, in order to receive a more detailed feedback on whether the flaws, detected in our first draft, were correctly fixed. This did not lead to a bias of our evaluation, thus the design of the mental and practical tasks were completely different than in the first draft. Fortunately, both, layout and structure of our prototype were accepted well. \\

We timed the interaction for each user in a specific manner, using a stopwatch. Our intention was to ensure that the accomplishment of the mental tasks, designed for long orientation time, truly took longer than the ones intended for short orientation time. The same was important for the practical tasks and their corresponding input times. We wanted measure the phases similar to the approach made by Anonymous et al. \cite{anonymous}. To do so, we determined a starting and ending point for each of both, mental and practical task, by setting a clear time interval for each of the phases, orientation and input. The mental task began as soon as a particular grid was uncovered and ended as soon as the hidden pattern was found. The find was indicated by placing a finger on the very first button of the pattern. The practical task began with the "first button press" and ended with the very "last button press". Although the first button had already been pressed to signify the find, it had to be pressed once a again for the input. We were aware that the measured times would not be completely accurate. They were only intended to serve as a rough estimate of the phases. We considered the first and last combination of each ratio to be an exercise for the participant to get acquainted with the concept. Therefore, we only considered the average times of the third and last combination of each ratio. Although not accurate, the time approximate duration resulted as we had imagined and desired for our concept.

\section{Implementation of the Concept} \label{4.2.2.5}

- 

LATER IN IMPLEMETNATION ! 
We also decided to shuffle the buttons inside each grid by hand, in order to have the freedom to build, so called traps that assure that finding the pattern wasn't too easy. In Figure [ADD IMAGE] there are couple example for some of our traps. Another reason for manually designing the grids, was to assure that the pattern was only contained once in the grid. We were not sure if both of our visions would have been possible, if we had used randomization to shuffle the buttons inside grids. \\

