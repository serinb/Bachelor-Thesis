
%*****************************************
\chapter{Developing an Interaction System}\label{ch:forth}
%*****************************************


Before entering into the design process it is crucial to mention that the following interaction system was created solely to serve as a tool in the user case study, which we will present in Chapter 5. The system is in no means a suggestion for authentication concept and was not intended to be utilized as such. It's sole utilization and purpose was meant for the scope of this thesis and no further. Therefore certain aspects such as security and effectiveness have not been considered during the design and the development of this system. Nonetheless, during its creation, we tried our best to follow the steps of a User-Centered Design (UCD) approach and included a selection of Human Computer Interaction (HCI) principles, in order to make the system easier on the human eyes and on the mind. It was important to make sure that the design and structure choices, neither distracted the user from the important factors of the study, nor complicated the effort needed to concentrate on them.  

\section{Requirements of the Concept}
As mentioned above, the interaction system was intended to be utilized to help examine certain factors later on in the study. These factors were \textbf{orientation time} and \textbf{input time}, previously explained in Chapter 3. In order to do so, the interaction system had to be based on a certain concept. This concept had to be divided into two coherent and related miniature tasks: a mental task and a practical task. The intention behind this division is to measure the overall time needed for the accomplishment, thereby distinguishing between the duration of the mental task (orientation time) and the duration of the practical task (input time). \\

In order to validate the findings presented in the previous chapter, we had to find a way through which we could examine the effect of orientation time on authentication, with respect of the input time. Therefore we decided to analyse two contrasting time ratios: 
\begin{center}
    \textbf{Long} orientation - \textbf{Short} input \\
    versus \\
    \textbf{Short} orientation - \textbf{Long} input
\end{center} 

In contrast to the study by Anonymous et al. \cite{anonymous}, we wanted to directly examine the influence that orientation time has on users' acceptance of an authentication concept, by comparing both contrasting variations against each other, using only one concept, instead of three. Therefore the concept had to be malleable in a way, such that the time needed to accomplish the miniature tasks, could be modified by adjusting their degree of difficulty and complexity. We imagined the less difficult and the less complex the task, the less time is needed for its accomplishment, and vice versa. This idea applied the mental task, as well as the practical task. \\

Another crucial requirement of the design, was that it included a built-in timer, that would measure the duration of the tasks separately and thereby deliver us the orientation and input needed. Despite the mentioned requirements being crucial for the development of the interaction system, finding a suitable concepts to base it on, was equally as important for the purpose of its functionality. In the next section, we will present the many design and evaluation attempts that were necessary to generate \underline{\textbf{FiPa}}: the interaction system presented and utilized in our study. 

\section{Concept Development}

\subsection{Fundamental Concept Idea}
We were interested in creating a concept that activated an interaction which believably resembled a smartphone authentication mechanism. We believed that by making the interaction similar to an authentication process, we could help our future participants better understand and adapt to the context of the study. In this way, we could receive more focused and detailed qualitative results. One of our concerns was to construct the concept as usable as possible, in order to shine the light on the true usability issues which we wanted to examine, being the orientation and input time ratios. We, therefore, decided to create a graphic-based concept. Although recent research has shown them to require longer to authenticate and to be more error-prone, it was found that they had a greater user-acceptance and a better ease-of-use, than other approaches. \\
We will now explain the basic idea for our concept: First a pattern is presented that has to be memorized well. This pattern consists of a certain combination of buttons. Each button has a certain characteristic, that makes it distinguishable from other buttons. Memorizing the order of the buttons and their distinct characteristics is crucial for the both mental and practical tasks. After memorizing the pattern, a large grid, filled with buttons, is presented. The mental task thereby is to search for the memorized pattern, hidden in the grid. When found, the practical task is to enter the found pattern correctly, as memorized. [REFER TO IMAGE HERE] Our idea was inspired by the playfulness of the concept Marbles [ADD SOURCE] and by the complexity of the concept Pattern Rotation [ADD SOURCE]. \\

ADD RECOGNITION RATHER THAN RECALL 

\subsection{Concept Design}
A couple of design ideas that were created, tested and evaluated and redesigned 

\subsubsection{First Draft}
Our initial vision for the design is presented in [ REFER TO EMOJI IMAGE HERE]. As in Marbles, we decided to make the different elements of the concept (the buttons) distinguishable through small images, emojis to be exact. We will first begin by explaining how we visualized the practical task and then proceed with explaining the mental task. As mentioned in Section 4.2.1, a certain pattern is required to be memorized at the very beginning of the activity. To mark the beginning of a pattern we determined that the first button of each pattern should contain a key-emoji [REFER TO IMAGE ]. Our intention for choosing the key-emoji will be comprehensible when we  explain our idea for the mental task. In order to modify whether the intended input time should be short or long, we created examples of how we imagined the patterns to look like in Figure [ADD IMAGE HERE]. As for the mental task, we envisioned to present a grid filled with emoji-buttons. The design of the grid depended on whether long orientation time or short orientation time was intended. As mentioned earlier in the chapter, we concluded that a long orientation time requires a difficult and complex search process. In order to ensure a short orientation time, we imagined it would be possible to facilitate the pattern search, by having the only key-button in the grid be the one belonging to the hidden pattern. in contrast, we imagined that by incorporating multiple key-buttons, besides the one belonging to the hidden pattern, we could overwhelm the user and thereby complicate the search process. 