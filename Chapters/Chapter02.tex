
%*****************************************
\chapter{Theme}\label{ch:second}
%*****************************************

The purpose of this chapter is to fill the gap of knowledge, by contextualizing the importance smartphone authentication and by illustrating why it is crucial and necessary to improve it. Next, we will introduce the research field of Usable Security by explaining their aim and the steps they are taking in order increase usability in security mechanisms, in general. Last but not least, we will present the recent work and appproaches made in the field of improving usability in smartphone authentication.   

\section{Context}

\subsection{Importance of Smartphone Authentication}

Nowadays, smartphones are approved to be one of the main essentials in people's day to day lives. Besides serving as a medium for basic communication, mobile phones have evolved in a remarkable way, offering advanced features and functions which were formally known to only be possible on a personal computer \cite{Alsaleh}. Besides storing and granting access to personal photos, videos, emails, and social media, smartphones enable their users to do money transfers, online shopping, and even track their own health through the tips of their fingers \cite{Egelman:2014:YRL:2660267.2660273,Albayram:2017:BUL:3235924.3235929,Schloeglhofer}. \\

Being the powerful and capable devices that they are, it is said that smartphones have the potential to replace the need for a personal desktop \cite{Alsaleh}. Hence, they should be capable of protecting their users' sensitive and private data confidentially and securely. The fact that users carry their mobile phones with them wherever they go, causes a threat that the devices might get  lost or even stolen \cite{Egelman:2014:YRL:2660267.2660273}. An American software company, named \textbf{Symantec}\footnote{https://www.symantec.com/de/de - last accessed: 2019/11/04}, conducted an experiment, where they purposely "lost" fifty unprotected smartphones in five destinations. They did so, to observe how the finders of the devices would behave and how the data stored on the devices would be treated. Surprisingly, they found that the data was accessed on 96\% of the smartphones and that only half of the finders offered to return the devices \cite{symantec}. It is safe to assume that a smaller amount of the data would have been compromised, had the phones been protected by a security mechanism. Considering the risk portrayed through the experiment and the threat of hackers gaining unauthorized access to private data, it is noticeable how important authentication mechanisms (\textbf{"screen locks"}) are for users' data security and privacy.\\

The purpose of an \textbf{authentication mechanisms} is to allow a person access to a certain medium, only after verifying and approving that they truly are who they claim to be. There are many ways in which this could take place. In general the mechanism asks the person, who is requesting access, to enter a secret which only the owner of the medium should know. It serves as a countermeasure to exploitation of personal and confidential information. There are many types of authentication mechanisms and these can be categorised as follows \cite{gorman}:  
\begin{itemize}
    \item \textbf{Biometric} - describes "what You are" e.g., fingerprint scanner and facial recognition.
    \item \textbf{Knowledge-Based} - describes "what You know" e.g., pattern, password, pin.
    \item \textbf{Token-Based} - describes "what You have" e.g., cryptographic key or chip.
\end{itemize}


When speaking of authentication in smartphones, there are a couple of well-known mechanisms that come to mind. These can also be categorized as follows \cite{ediss20251,gorman} : 
\begin{itemize}
    \item \textbf{Alphanumeric} e.g., password and pin
    \item \textbf{Gesture-Based} e.g., Android Unlock Pattern
    \item \textbf{ID-Based} e.g., fingerprint scanner and facial recognition 
\end{itemize}

To this day, researchers have been working on improving authentication in smartphones and developing new concepts and designs for it. One might wonder, why this is necessary, when there are a hand full of methods already available. Are the current mechanisms not sufficient or secure enough? The answer to this question is twofold:\\

On the one hand, certain authentication mechanisms lack in security and are vulnerable towards certain attacks \cite{Schloeglhofer}. For instance, Android Unlock Pattern is known to be vulnerable towards so called \textbf{smudge attacks}. These are attacks that occur when a victim has previously drawn their secret pattern, leaving an oily trace of their finger on the touch screen. These smudge marks help the attacker to guess the secret pattern, bypass the security measure, and gain access to the device \cite{ediss20251}. Another popular attack, which occurs especially "in the wild" are \textbf{shoulder surfing attacks}. These happen when the victim attempts to authenticate themself in public and uncontrolled surroundings. If the attacker is situated in a near distance, behind or beside the victim, they are able to observe the input of the secret and memorize it for later use \cite{ediss20251}. Authentication mechanisms, such as pin, password and Android Unlock Pattern are prone to such threat, especially when the alphanumeric or pattern secrets are short and simple enough to memorize easily. There have been many research contributions which propose improvements of existing mechanisms as well as the development of new ones to counteract these security threats. Zezschwitz et al. \cite{vonZezschwitz:2015:SFS:2702123.2702212} developed an interesting authentication concept called \textbf{SwiPIN}, intended to protect pin authentication from shoulder surfing attacks. Another proposal, \textbf{Tinylock}, made by Kwon et al. \cite{kwon}, acts against both smudge and shoulder surfing attacks. \\



On the other hand, certain authentication mechanisms still lack in user-friendliness, and are therefore not as usable as their developers intend for them to be \cite{Schloeglhofer}. As a result, research has shown that many users consciously choose not to use an authentication mechanism on their smartphones \cite{ediss20251, Albayram:2017:BUL:3235924.3235929, Egelman:2014:YRL:2660267.2660273}. Studies have indicate that one of the main reasons for such behaviour is that users perceive screen locks as an inconvenience \cite{Albayram:2017:BUL:3235924.3235929, ediss20251, harbach}. Another reason was found to be a lack of knowledge about smartphone security \cite{Albayram:2017:BUL:3235924.3235929, Adams:1999:UE:322796.322806}. Researchers discovered that some users opt to not install a screen lock, because they underestimate the risk that comes with not having one and because they do not comprehend to which extent their data is at stake \cite{Egelman:2014:YRL:2660267.2660273}. Through finding out this information, investigations were made on how to create authentication mechanisms that aren't just secure, but also usable. On that note, studies have been made on examining authentication from the user's perspective and finding out ways to create concepts that satisfy both sides: \textbf{usability} and \textbf{security}. 


\subsection{Improving Usability in Smartphone Authentication}

As mentioned earlier, security, alone, is not sufficient enough to guarantee the true success of an authentication mechanism. A lack of usability in security mechanisms defeats their purpose, no matter how secure they are in theory. \textbf{Usable security} is a young and widely popular research field who's aim is to create a balance between \textbf{usability} and \textbf{security} in security systems and mechanisms \cite{Realpe-Munoz, anonymous}. In order to better understand their ambitions, we will first give an understanding on what usability is. \\

\textbf{Usability} generally describes the degree to which a user is able to accomplish a certain task with "effectiveness, efficiency and satisfaction" when utilizing a certain product.\footnote{https://www.interaction-design.org/literature/topics/usability - last accessed: 2019/11/10} While this definition applies to all designed and developed products imaginable, it certainly also applies to security mechanisms. The goal of usable security experts is to construct the design process for a security measure similar to the design for any product intended for human use. In other words, security designers should implement a user-centered approach when designing security mechanisms in order to involve certain human factor principles, crucial to their success \cite{Adams:1999:UE:322796.322806, sasse}. Interestingly, the importance of usability was initially established in 1975 in a research article titled \textit{"The Protection of Information in Computer Systems"}. To our knowledge, the relevance of user-centered design was first introduced nearly two decades ago in an article called \textit{"Users are not the enemy"} \cite{Adams:1999:UE:322796.322806}. \\

Despite previous discoveries, usability of authentication mechanisms still remains a conundrum waiting to be solved. While users find it hard to comply with required security guidelines and therefore behave insecurely \cite{Adams:1999:UE:322796.322806, sasse}, security experts perceive users as \textit{"the weakest link in the chain of system security"} \cite{sasse} and find that they are \textit{"a security risk that needs to be controlled and managed"}  \cite{Adams:1999:UE:322796.322806}. Hackers have learned how to benefit from this situation, by using social engineering methods to obtain individuals authentication secrets \cite{Adams:1999:UE:322796.322806, sasse}. They notice the flaws in current security systems and are able to foresee how users will behave. Users, however, are not to fully to blame for this issue. The reason why hackers are able to attain unauthorized access to systems so easily is because they are more attentive to users' perception of security than security designers are \cite{Adams:1999:UE:322796.322806}. By nature, humans are prone to try and find shortcuts and time-saving methods when it comes to difficult tasks \cite{sasse}. This applies especially when using security mechanisms which demand actions that are either impossible or unnatural to follow \cite{sasse}. For instance, when using a password protected system, requirements are to use an alphanumerical secret that is at least eight characters long, consists of lower and upper case letters as well as special characters \cite{payne, sasse}. Furthermore, passwords are required to be changed regularly \cite{adams2,gorman}. These regulations might be easy to comply with when the user only has one password to memorize. However, it is evident that nowadays users have to manage a multitude of passwords, which makes following the guidelines more tiresome and difficult. To that end, users are bound to seek a solution that helps them bypass security measures in order to work on the task they initially intended to achieve. \\

Experts in human factors differentiate between two types of tasks: \textbf{productive tasks} and \textbf{supportive tasks} \cite{sasse}. Productive tasks are the activities needed to accomplish a certain goal or to reach a certain outcome \cite{sasse}. They fulfill the purpose of a system's existence \cite{sasse}. Supportive tasks are ones intended to help productive tasks in being executed efficiently and on a permanent basis \cite{sasse}. According to Sasse et al. \cite{sasse}, security mechanisms are considered to be supporting tasks. The grave problem, thereby, is that often times the requirements of current security mechanisms contradict or do not match the demands of the tasks which they support. In turn, the efficiency of the production task is reduced due to the delay of its accomplishment. The reason for this, is that, in general, security mechanisms are not coherent with the demands and needs of productive tasks. Therefore, users are involuntarily put in a position to choose between which task to prioritize. Since the production task generally is their initial goal, they find themselves looking for ways to bypass or neglect security systems. \\

In order to find a solution for this matter, each form of security measure has to be researched separately, so that the mechanisms can be specifically made to adapt the context of the productive tasks that they support. Thus the main focus of this thesis is smartphone authentication, the following we
presenting certain approaches and suggestions made towards creating efficient and user-friendly authentication mechanisms for smartphones. Thereby the following research questions will be of interest: 


\begin{itemize}
    \item \textit{What are the factors that affect usability in smartphones authentication?}
\item \textit{How can we evaluate whether an authentication mechanism is usable, or not?} 
    \item \textit{How can we compare multiple systems based on their usability and evaluate them accordingly?}
\end{itemize}


\section{Related Work}

OUTLINING
\begin{itemize}
    \item behaviour and perception
    \item methods of intervention
    \item analysis and comparison of current mechanisms
\end{itemize}

In the following section we will discuss a collection of approaches made in the field of \textbf{Usable Security} with the intention of analysing smartphone authentication from the users' perspective and thinking of new ways to improve security mechanisms in terms of their efficiency and effectiveness \cite{anonymous}. \\

In general when solving the matter of usability in security systems we cannot expect to find a sole solution that will solve all usability issues. Thus it is a very complex and broad subject and there is a wide range of factors that have been found to play a role in increasing or decreasing usability in smartphone security[add sources]. Researchers have made the effort of finding the reason for users' insecure behaviour by conducting studies to observe their attitude towards certain authentication mechanisms and smartphone security, in general.[add source] \\

Are the following references enough? 

\begin{itemize}
    \item refer to : 
    \item "better ..." by Albayram 2017
    \item Keep on lockin 2016
    \item its a hard lock life 2014
    \item understading how security mechanisms are perceived 2017
    \item the anatomy of smartphone unlocking 2016
    \item patterns in the wild 2013
    \item are you ready to lock ?
    \item emanuel diss on mapping and design and usability , temporal arrangement 
\end{itemize}






