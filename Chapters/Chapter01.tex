%************************************************
\chapter{Introduction}\label{ch:first}
%************************************************

\section{Motivation}

\section{Outline}


\begin{itemize}
    \item in this day and age data security and data privacy has become a major and important. 
    \item Up until the internet was invented and computers and mobile devices are not only used by employees of major companies or organization, who are trusted with their secret information and databases, data security has gotten a new definition. 
    \item While, back in the day, it was common to have your private Information safe and sound at home, meaning your photos, letters, bills, etc. , people wouldn't worry much about strangers physical access to their private belongings, unless that stranger would brake into their house. 
    \item yet nowadays, private information is stored in so called clouds on the internet, to which one is only granted access, if one correctly enters a certain secret. 
    \item With the usage of smart phones and the fact that almost every person owns one, the potential of being a victim of a cyber attack is even higher
    \item Since Smart Phones are so widely developed that they can be used to work on tasks that one would normally do on a computer 
    \item They do not only contain private belongings, such as photos and message conversation, but also enable access to bank transaction and SOMETHING ELSE ! 
    \item On that note, cyber theft has become a serious issue for every Smart Phone User.
    \item The Most commonly used knowledge-based Smart Phone authentication are PIN, Password and Pattern, which are used for Android Phones, although one might say, that they are easy to use, many users choose to omit setting an authentication mechanisms for their phone, because it's time consuming [set quote here]
    \item Furthermore they are sensitive to certain security attacks such as shoulder surfing or smudge attacks 
    \item also users that have a password lock on their smartphone tend to either use simple secrets, which do not suffice the required password guidelines or they reuse one of their already existing passwords. This, of course leads to an even bigger hazard, given the attacker possesses the needed resources to gain access to the other accounts that the password was used for.
    \item Although biometric authentication such as fingerprint or facial recognition, has become quite popular, the user has to set one of the earlier mentioned methods as a fallback method.
    \item Since the recent mechanisms are vulnerable to attacks, researchers have tried to design and develop new authentication mechanisms that more secure and resilient against the mentioned attacks, such as [Mentin some Concepts]
    \item Each newly developed concept or idea for an authentication mechanisms undergoes a process of examination in form of a user case study, investigate whether or not the new design delivers what it's developers promise it does.
    \item In this user case study, three main aspects of the app are usually measured and analysed, which are efficiency and security, usability.
    \item While measuring security and efficiency has a pretty straight-forward process, usability is usually qualitively measured, either through a questionnaine in an interview
    \item However there is no generally used standard for measuring usability. This arises a couple of questions, such as,
    \item how do we know if a certain developed concept truly is usable 
    \item What are certain aspects that make some authentication mechanisms more usable than others? 
    \item How can we compare mechanims with each other, in order to find out which one is more usable than the other? 
    \item in the paper (...) researchers "present a time measurement approach", which is intended to help develop more usable authentication mechanisms for smartphones in the future
    \item the time measurement is based on many different aspects of mechanism which were found to have an influence on the usability of the concept
    \item continue when You have written about the paper in RELATED WORK !! 
    
    
    
    
    
    
    
    IMPORTANT !!!!
    
    - it is common that in order to measure efficiency, the input speed of the authentication is measured
    - this always leads to the fact that measured efficiency is often times different than perceived efficiency
    - but although an input time is short, why might users perceive it as long? 
    - because the long orientation 
    - usability is a priority, but the main focus here is efficiency
    
    
    
    
    
    
    
    
    
    
    
    
    
    
    \item Overview of the Thesis
\end{itemize}




