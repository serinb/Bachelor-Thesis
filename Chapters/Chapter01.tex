%************************************************
\chapter{Introduction}\label{ch:first}
%************************************************

Nowadays, smartphones are approved to be one of the main essentials in people's day to day lives. Besides serving as a medium for essential communication, mobile phones have evolved remarkably, offering advanced features and functions which were formally known to only be possible on a personal computer \cite{Alsaleh}. Besides storing and granting access to private photos, videos, emails, and social media, smartphones enable their users to do money transfers, online shopping and even track their health through the tips of their fingers \cite{Egelman:2014:YRL:2660267.2660273,Albayram:2017:BUL:3235924.3235929,Schloeglhofer}. \\

Being the powerful and capable devices that they are, it is said that smartphones have the potential to replace the need for a personal desktop \cite{Alsaleh}. Hence, they should be capable of protecting their users' sensitive and private data confidentially and securely. The fact that users carry their mobile phone with them wherever they go causes a threat of their device getting lost or even stolen \cite{Egelman:2014:YRL:2660267.2660273}. An American software company, named \textbf{Symantec}\footnote{https://www.symantec.com/de/de - last accessed: 2019/11/04}, conducted an experiment, where they purposely "lost" fifty unprotected smartphones in five destinations. The company did so, to observe how the finders of the devices would behave and how they would treat the data stored on these devices. Surprisingly, they found that the data was accessed on 96\% of the smartphones and that only half of the finders offered to return the devices \cite{symantec}.  This experiment portrays a security risk that is likely to affect any smartphone user.  It is safe to assume that a smaller amount of the data would have been compromised, if the phones had been protected by a security mechanism. Therefore, it is evident that securing smartphones with an authentication mechanism is vital for users' data security and privacy.

\section{The purpose of authentication mechanisms}

The purpose of an \textit{authentication mechanism} is to allow a person access to a particular medium, only after verifying and approving that they indeed are who they claim to be. There are many ways in which this action could take place. In general, an authentication mechanism asks the person, who is requesting access, to enter a secret which only the owner of the medium should know. It serves as a countermeasure to the exploitation of the owner's personal and confidential information. There are many types of authentication mechanisms, and these can be categorised as follows \cite{gorman}: 
\begin{itemize}
    \item \textbf{biometric} - describe "what You are" e.g., fingerprint scanner and facial recognition.
    \item \textbf{knowledge-based} - describe "what You know" e.g., pattern, password, pin.
    \item \textbf{token-based} - describe "what You have" e.g., cryptographic key or chip.
\end{itemize}

When speaking of authentication in smartphones, there are a couple of well-known mechanisms that come to mind. These can also be categorized as follows \cite{ediss20251,gorman} : 
\begin{itemize}
    \item \textbf{alphanumeric} e.g., password and pin
    \item \textbf{gesture-based} e.g., Android Pattern Unlock
    \item \textbf{ID-based} e.g., fingerprint scanner and facial recognition 
\end{itemize}

To this day, researchers have worked on improving smartphone authentication by developing new concepts and designs for them. One might wonder why this is necessary when there is a decent collection of methods already available. Are the current mechanisms not sufficient or secure enough? \\

The answer to this question is twofold:\\

On the one hand, specific authentication mechanisms lack in security and are vulnerable towards certain attacks \cite{Schloeglhofer}. For instance, Android Pattern Unlock is known to be vulnerable towards so-called \textbf{smudge attacks}. These are attacks that occur when the victim draws their secret pattern, leaving an oily trace of their finger on the touch screen. These smudge marks help the attacker to guess the secret pattern, bypass the security measure, and gain access to the device \cite{ediss20251}. Another popular attack, which occurs primarily "in the wild" are \textbf{shoulder surfing attacks}. These happen when the victim attempts to authenticate themself in public and uncontrolled surroundings. If the attacker is situated in a near distance, behind or beside the victim, they can observe the input of the secret and memorize it for later use \cite{ediss20251}. Authentication mechanisms (e.g., pin, password, and Android Pattern Unlock) are prone to such threats, especially when the alphanumeric or graphical secret is short and simple enough to memorize easily. There have been many research contributions that proposed improvements for existing mechanisms as well as the development of new ones to counteract these security threats. Zezschwitz et al. \cite{vonZezschwitz:2015:SFS:2702123.2702212} developed an authentication concept called \textbf{SwiPIN}, intended to protect pin authentication from shoulder surfing attacks. Another proposal, \textbf{TinyLock}, made by Kwon et al. \cite{kwon}, acts against both smudge and shoulder surfing attacks\footnote{There are many solutions to these attacks other than TinyLock and SwiPIN. In Section \ref{2.2.3}, we will only present SwiPIN as an example for all, to not exceed the scope of this thesis.}.\\

On the other hand, certain authentication mechanisms still lack user-friendliness and are therefore not as usable as their developers intend for them to be \cite{Schloeglhofer}. As a result, research has shown that many users consciously choose not to use an authentication mechanism on their smartphone \cite{ediss20251, Albayram:2017:BUL:3235924.3235929, Egelman:2014:YRL:2660267.2660273}. Studies have indicated that one of the main reasons for such behavior is because users perceive screen locks as an inconvenience \cite{Albayram:2017:BUL:3235924.3235929, ediss20251, harbach}. Another reason was found to be a lack of knowledge about smartphone security \cite{Albayram:2017:BUL:3235924.3235929, Adams:1999:UE:322796.322806}. Researchers discovered that some users decide not to install a screen lock, because they underestimate the risk that comes with not having one and because they do not comprehend to which extent their data is at stake \cite{Egelman:2014:YRL:2660267.2660273}. Through these findings, investigations were made on how to create authentication mechanisms that are not only secure but also usable.

\section{Thesis Aim and Outline}

In an effort to enhance the usability of smartphone authentication mechanisms, researchers have proposed a selection of novel authentication concepts. Formerly, they would evaluate the usability of a concerning authentication concept, by solely measuring its \textit{input time}, meaning the time needed to enter a particular secret (e.g., password, pin, pattern) \cite{Zezschwitz}.
However, recent research has shown that there are more factors, other than \textit{input time}, that affect the usability of an authentication concept. The most effective factor was found to be \textit{orientation time}, which is the time that a user spends thinking about the secret prior to entering it \cite{Zezschwitz}. Zezschwitz et al. \cite{Zezschwitz} made an effort to examine the effect of \textit{orientation} and many other factors more closely by redefining the structure of a general authentication process and by dividing it into multiple phases, including the \textit{orientation} and \textit{input}. Through a user case study, they tested the validity of their approach by analyzing three authentication concepts, each representing a different representation (ratio) of the \textit{orientation} and \textit{input phases}. They discovered that the efficiency, more specifically, the perceived efficiency of an authentication concept is crucial for determining how usable it is. Moreover, they found that perceived efficiency is mostly impacted by the representation of the \textit{orientation phase}. \\

Our thesis aims to complement the findings of Zezschwitz et al. \cite{Zezschwitz} by testing their approach, by using a different technique. Instead of testing the different variations of \textit{orientation} and \textit{input time}, by analyzing different concepts, we specifically developed \underline{\textbf{FiPa}}, a concept representing all studied variations. Next, a complementary user case study was conducted with the purpose of proving that \textit{orientation time} indeed causes the effects, discovered by Zezschwitz et al. \cite{Zezschwitz}.\\

The thesis will be organized as follows: \\

Chapter \ref{ch:second} will contextualize the improvement of smartphone authentication in the research field of \textit{Usable Security}, by first discussing their aim and ambitions. Then a set of selected related work will be presented, to illustrate recent approaches that have been taken to improve the problem of usability in smartphone authentication. Lastly, the main focus of this thesis will be extracted from these findings, by discussing and outweighing their true potential and effect. \\

Chapter \ref{ch:third} will illustrate the theoretical groundwork of our thesis, which is the approach and the findings of the ongoing study conducted by Zezschwitz et al. \cite{Zezschwitz}. In the end of the chapter, the limitations of their work will be pointed out and balanced by a selection improvement propositions which were implemented in our contribution to this thesis.\\

Chapter \ref{ch:forth} will introduce the concept \underline{\textbf{FiPa}}, which was intentionally designed and developed to serve as a tool in the user case study, presented in Chapter \ref{ch:fifth}. Although it was solely developed for the sake of our study and was not intended for any further use, we involved a series of HCI principles and insisted that its development process resembled that  of a user-centered design approach, as, nevertheless, it was intended for human use.\\

Chapter \ref{ch:fifth} will present the user case study, which we conducted to validate the findings, presented in Chapter \ref{ch:third}.
It thoroughly illustrates the design, procedure and results of our study.\\

Chapter \ref{ch:sixth} will discuss our results, presented in Chapter \ref{ch:fifth} and will analyze whether our research truly complements the findings and observations, presented in Chapter \ref{ch:third}. In addition, the chapter will elaborate on the limitations which we faced along the way. \\
 
Lastly, Chapter \ref{ch:seventh} will provide a conclusion regarding our findings  and will present a set of possible future work propositions, which may help in achieving further progress in the enhancement of usability in smartphone authentication.






