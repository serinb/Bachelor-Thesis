
%*****************************************
\chapter{Case Study}\label{ch:fifth}
%*****************************************

Our ambition in this part of the thesis, was to test the concepts in the interaction system, which were earlier introduced in Chapter 3, in terms of their perceived efficiency and to analyse their user acceptance.



INTRODUCE THIS CHAPTER AND EXPLAIN WHAT WILL BE DISCUSSED IN EACH OF THE SECTIONS \\

We conducted a user case study, in form of a LABOR STUDIE in order to validate the findings in paper [zezeschwitz]. Their findings state that orientation time has a major influence on a user's general preference of authentication systems, based on their usability aspect. In other words, long orientation periods make the overall authentication time appear longer, which in return decrease the perceived efficiency of a authentication mechanism.\\

[Zezschwitz] were able to present remarkable results through their research. They made their examinations by asking participants to use three different authentication concepts, each for a period of ten days. These, were the following: 

\begin{itemize}
    \item Android Pattern Unlock, which represents the "Short Orientation / Short Input" ratio
    \item Pattern Rotation, which represent the "long orientation / short input" ratio
    \item Marbles, which ration, where the orientation and input times are intertwined.
\end{itemize}

Through their study, they showed that users are more likely to prefer authentication systems that have a short orientation time which, therefore, appear less mentally demanding - As already mentioned in [CHAPTER].\par
In order to examine the validity of these findings even closer, we developed an interaction system as an application [CHAPTER REFERENCE] to study the overall affect of orientation time regarding perceived efficiency.  by analysing the following ratios: 

\begin{itemize}
    \item short orientation / short input, which was used as a baseline for the time measurement
    \item short orientation / long input (mainly examined phases)
    \item long orientation / short input
\end{itemize}

Since the study, conducted by [Zezschwitz], did not include the case for "Short Orientation / Long Orientation", we found it interesting to find out how the users would react to both main ratios, in comparison to each other. We did so, by presenting an interaction mechanism [READ CHAPTER] which emulated an authentication concept, presenting each of the mentioned ratios. In other words, we intended for our examination to be more focused and exact, by narrowing down the independent variable from "authentication concept" to "orientation and input time ratio". 


\section{Methods}


\subsection{Design}

\begin{itemize}
    \item die Reihenfolge in der die Verfahren getestet werden sollten, wurden unter den Teilnehmern aufgeteilt - PERMUTATION ZEIGEN
    \item 
\end{itemize}




\begin{itemize}
    \item TYPE OF STUDY 
    \item The only independent variable was orientation and input time ratio which had three levels:
    
    \begin{enumerate}
        \item short orientation / short input
        \item short orientation / long input
        \item long orientation / short input
    \end{enumerate}
    
    \item the duration of the study was 15 minutes on average. 
    \item the order of the ratios in the application, were COUNTERBALANCED BETWEEN THE PARTICIPANTS [zezschwitz]
    \item In order to measure the efficiency, we collected quantitative and qualitative data. We did so, by measuring the each orientation and input time during the interaction and storing the data in a local database, and by asking efficiency related questions in a questionnaire.
\end{itemize}



\subsection{Participants}

Twenty-five participants were recruited for the study, of which seventeen (62\%) were male and eight (32\%) were female. The average age was 22 years. Initially, we recruited a part of the participants online, through a online platform for computer science students of the Rheinische Friedrich-Wilhelms Universitaet Bonn. Another part was collected on campus of the university's computer science department, and four participants were acquaintances of the author. There was no premise for participating in the study, that means anyone was eligible to partake.



\subsection{Procedure}
The study was held for each participant separately. We began by handing out a declaration of consent, which stated the purpose of the study and guaranteed to treat the data, collected during the study, discretely and anonymously. After assuring that the participant agreed on the consent, by signing it, we then proceeded by explaining the main purpose of the study:\par

\textbf{To analyse certain factors in smartphone authentication that might play a role in its perceived efficiency.}\par

We emphasized that the purpose of the study is to solely to improve usability of authentication mechanisms, and we clarified that the security aspect was outside of the scope of our research. Furthermore, we assured that the purpose of the test, is not to evaluate their performance or intelligence. It was important for us that the participant did not feel put under pressure or that they were under observation, whilst undergoing the test, in order to ensure that the environment and atmosphere was as close to a real-world scenario, as possible.\
Next, we described the structure of the application. We defined it as an interaction system which emulates an authentication concept. Furthermore, we mentioned that the application presented a series of small challenges, for them to solve. These simple challenges were explained with the help of a paper prototype.\\

The participant was able to test the application, after ensuring that they had understood the procedure of the study and that they had no further questions. Before being introduced to the actual test, participants were first led through a training segment, which was also part of the application. It was crucial that they thoroughly understood the functioning of the application, and that they got accommodated with its use, in order to prevent their lack of understanding being a threat to the validity of the measured data. They were able to repeat the segment as often as they wanted, and were given the choice to start the test whenever they felt ready. We did our best to create a calm and friendly atmosphere around the participant, so that they did not feel nervous or anxious during the test. We guided them through the training segment, when help was needed, and took the chance to explain to them the certain rules of the challenges, such as :

\begin{itemize}
    \item RULES !!!
\end{itemize}

To start the test, we first entered a user-id, in order to be able to pair the participant's measured data with their questionnaires. we did not intervene, whilst they were working on the test. Once they were done, they were given a questionnaire to fill out. After the study, the each participant was compensated with 5 Euros.

\subsection{Instruments}

\subsection{Constraints}

\subsection{Results}
HERE !!! 
