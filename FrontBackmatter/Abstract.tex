%*******************************************************
% Abstract
%*******************************************************
%\renewcommand{\abstractname}{Abstract}
\pdfbookmark[1]{Abstract}{Abstract}
\begingroup
\let\clearpage\relax
\let\cleardoublepage\relax
\let\cleardoublepage\relax

\chapter*{Abstract}
 This thesis represents a complementary research contribution to an ongoing study, which is concerned with improving the usability of smartphone authentication mechanisms. More specifically, the improvement of the perceived efficiency of authentication concepts is of interest. Certain factors which define the nature of an authentication process have been found to have an impact on the perceived efficiency of a concept. One of the main factors is orientation time, which defines the time interval in which a user prepares themself for the input. Another crucial factor is input time, which defines the time a user needs to enter a secret when authenticating. The ongoing study has delivered a collection of observations regarding the effect of orientation time on perceived efficiency. One of which, was that users generally perceive authentication concepts which have "long orientation times" as not usable. More specifically, concepts whose orientation time exceeds the corresponding input time are highly considered as inefficient. To validate these observations, we implemented a similar study approach combined with a set of suggestive improvements to increase the accuracy of our results. We specifically developed a concept called \underline{Fipa} (\underline{Fi}nd \underline{Pa}ttern) to represent a selection of orientation and input time ratios, intended for evaluation and analysis in a user case study. We confirm and validate previous observations and findings by showing that users generally prefer "short orientation" over "long orientation time". In comparing quantitative and qualitative results, we also show that human perception is more sensitive towards orientation times than it is towards input times. We imply that users are more accepting of "long input times" than they are of "long orientation times", as they have shown to be less bothered by them. We confirm that through analyzing measured and perceived efficiency of authentication concepts, we can come one step closer to understanding the human interpretation of what is efficient and create authentication concepts that are not only secure, yet also natural and suitable for human use. 

\vfill

\pdfbookmark[1]{Zusammenfassung}{Zusammenfassung}
\chapter*{Zusammenfassung}
Die folgende Thesis stellt einen erg{\"a}nzenden Forschungsbeitrag zu einer laufenden wissenschaftlichen Arbeit dar, welche sich mit der Verbesserung der Benutzerfreundlichkeit von Authentifizierungsmechanismen bei Smartphones befasst. Insbesondere ist die Verbesserung der wahrgenommenen Effizienz von Authentifizierungskonzepten von Interesse. Es stellte sich heraus, dass bestimmte Faktoren, die die Struktur eines Authentifizierungsprozesses definieren, eine Auswirkung auf die wahrgenommene Effizienz eines Konzepts haben.
Einer der Hauptfaktoren ist die Orientierungszeit (engl. orientation time), welche das Zeitintervall definiert, in dem sich ein Benutzer auf die Eingabe vorbereitet. Ein weiterer entscheidender Faktor ist die Eingabezeit (engl. input time), die die Zeit, die ein Benutzer ben{\"o}tigt, um eine Eingabe bei der Authentifizierung auszuf{\"u}hren, definiert. \\

Die bereits laufende Studie hat eine Sammlung von Beobachtungen zum Einfluss der Orientierungszeit auf die wahrgenommene Effizienz geliefert, z.B., dass Benutzer, im Allgemeinen, Authentifizierungskonzepte mit "langen Orientierungszeiten" nicht pr{\"a}ferieren. Insbesondere werden Konzepte, deren Orientierungszeit die dazugeh{\"o}rige Eingabezeit {\"u}berschreitet, als ineffizient angesehen. Um diese Beobachtungen zu validieren, haben wir den Ansatz unserer Studie auf den Ansatz der laufenden Studie aufgebaut. Dabei wurden einige Verbesserungs- und {\"A}nderungsans{\"a}tze in Betracht gezogen, um die Validit{\"a}t unserer Ergebnisse zu bekr{\"a}ftigen. Dazu haben wir ein Konzept, namens \underline{FiPa} (\underline{Fi}nd \underline{Pa}ttern), entwickelt, welches bestimmte Variationen von Orientierungs- und Eingabezeitverh{\"a}ltnissen darstellt. Diese wurde unterst{\"u}tzend in unserer Studie verwendet, um den Einfluss von Orientierungszeit auf die wahrgenommene Effizienz von Authentifizierungskonzepten zu untersuchen. Wir best{\"a}tigen und erkennen fr{\"u}here Beobachtungen und Ergebnisse als valide an, indem wir zeigen, dass Benutzer im Allgemeinen "kurze Orientierungszeit", "langer Orientierungszeit" vorziehen. Beim Vergleich quantitativer und qualitativer Ergebnisse stellen wir heraus, dass die menschliche Wahrnehmung gegen{\"u}ber Orientierungszeiten empfindlicher ist als gegen{\"u}ber Eingabezeiten. Wir vemuten, dass Benutzer "lange Eingabezeiten" eher akzeptieren als "lange Orientierungszeiten", da sie sie in unserer Studie, nachweislich, mehr pr{\"a}feriert haben. Wir best{\"a}tigen, dass wir durch die Analyse der gemessenen und wahrgenommenen Effizienz von Authentifizierungskonzepten der menschlichen Interpretation dessen, was effizient ist, einen Schritt n{\"a}her sind. Dadurch sind wir in der Lage Authentifizierungskonzepte zu erstellen, die nicht nur sicher, sondern auch nat{\"u}rlich und f{\"u}r den menschlichen Gebrauch geeignet sind.


\endgroup			

\vfill 