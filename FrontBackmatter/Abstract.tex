%*******************************************************
% Abstract
%*******************************************************
%\renewcommand{\abstractname}{Abstract}
\pdfbookmark[1]{Abstract}{Abstract}
\begingroup
\let\clearpage\relax
\let\cleardoublepage\relax
\let\cleardoublepage\relax

\chapter*{Abstract}
This thesis represents a complementary research contribution to an ongoing study, which is concerned with improving the usability of smartphone authentication mechanisms. More specifically, the improvement of the perceived efficiency of authentication concepts is of interest. Our goal was to validate and complement the findings and observations made in the ongoing study. We improved its approach by taking into account the limitations which were faced, and by adding further improvements and modifications to it. Based on a proposed dissection of the authentication process, it was observed that a certain part of the process, called orientation, has a significant effect on the perceived efficiency of an authentication concept. Orientation time is defined as the time a user spends preparing themselves to enter a secret, when authenticating. Another crucial part of the authentication process is input, which represents the time period in which a secret is entered. With the help of a specifically developed concept, intended to emulate an authentication concept, we represent certain ratios of orientation and input time. These ratios are then tested and evaluated in a user case study (n=19). Through a specifically designed quantitative and qualitative evaluation, we confirm the findings and observations made in the ongoing study and show that users generally dislike authentication concepts, in which orientation time is not only long, but also exceeds the duration of its corresponding input time. Moreover, the importance of taking the aspect of perceived efficiency into account, when evaluating the usability of an authentication mechanism, as it may deliver more specific and comprehensible information on smartphone users generally perceive and interpret to be efficient. 

\vfill

\pdfbookmark[1]{Zusammenfassung}{Zusammenfassung}
\chapter*{Zusammenfassung}
Kurze Zusammenfassung auf Deutsch


\endgroup			

\vfill 